%! Author = Frederik Bußmann
%! Date = 22.06.2023

\section{Entwicklung der CI-Strategie} \label{sec:04-implementation}

Im folgenden wird die konzipierte \acrshort{ci}-Strategie des vorherigen Kapitels praktisch umgesetzt.
Zunächst wird eine Projektumgebung geschaffen, in der die geplanten \acrshort{ci}-Tools und eine Shopware-Instanz zum
Testen installiert werden.
Daraufhin werden die installierten Tools für das Testen des lokal aufgesetzten Shops konfiguriert und angepasst,
wobei der Code eines eigens dafür angelegten Beispiel-Plugins als Test-Grundlage verwendet wird.
Anschließend werden verschiedene Deployment-Umgebungen für das Testen des Shops angelegt und für diese Umgebungen
automatisierte Deployment-Konfigurationen zum Ausliefern durch die zu entwickelnden Pipelines erstellt.
Zuletzt werden anhand der konfigurierten Tools und dem Beispiel-Shop die eigentlichen Pipelines zum
automatisierten Bauen, Testen und Ausliefern von Shopware-Projekten erstellt.

\subsection{Aufsetzen der Projektumgebung} \label{subsec:04-implementation-1}

% Installieren von CI-Tools und Shopware, Aufsetzen des VCS

Für das Implementieren der konzipierten Strategie muss zunächst eine lokale Entwicklungsumgebung bestehen, in der ein
Shopware-Projekt und die geplanten \acrshort{ci}-Tools installiert werden können.
Hierbei wurde sich für die Nutzung des von Shopware gestellten Docker-Containers entschieden, welcher die
Abhängigkeiten für das Betreiben der Plattform gesammelt bereitstellt.\footpartcite{shopware-docker}
Das Ausführen der Shop-Software ohne Docker ist dabei weiterhin möglich, indem dessen Abhängigkeiten direkt auf dem
Entwicklungssystem installiert werden.
Zunächst wurde ein Ordner für die Shopware-Installation angelegt und Package-Konfigurationen für Composer und NPM
erstellt.
In der Composer-Konfigurations-Datei\ \shellinline{composer.json} werden die Pakete für die PHP-basierten
\acrshort{ci}-Tools PHPUnit, Infection, \acrshort{phpcs}, \acrshort{phpmd}, PHPStan, Deptrac und License Checker
verwaltet.
Die Paket-Abhängigkeiten für die JavaScript-basierten Tools Eslint, Jest, Cypress und Danger JS werden in der
Konfigurations-Datei\ \shellinline{package.json} hinterlegt.

\subsection{Konfiguration von Testing- und QA-Tools} \label{subsec:04-implementation-2}

% Konfigurieren von Unit-, Mutation- und E2E-Tests + Static Code analysis, etc.

\subsection{Deployments und Umgebungen} \label{subsec:04-implementation-3}

% Hosting von Servern für Shopware-Umgebungen zum Testen aufsetzen und Deployer implementieren

\subsection{Implementierung der Pipelines} \label{subsec:04-implementation-4}

% Pipelines erstellen und konfigurieren

\clearpage

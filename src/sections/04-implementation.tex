%! Author = Frederik Bußmann
%! Date = 22.06.2023

\section{Umsetzung der CI-Strategie} \label{sec:04-implementation}

Im Folgenden wird die zuvor konzipierte\ \acrshort{ci}-Strategie in die Praxis umgesetzt und dabei in einer Fallstudie
exemplarisch auf ein Projekt angewandt.
Zunächst wird das ausgewählte Projekt vorgestellt, um einen Kontext für die anschließende Implementierung zu schaffen.
Dabei werden die spezifischen Anforderungen und Besonderheiten des Projekts hervorgehoben.
Im Anschluss daran wird die praktische Umsetzung der \acrshort{ci}-Strategie beschrieben, wobei sowohl die technischen
Aspekte als auch die organisatorischen Prozesse berücksichtigt werden.
Hierbei werden\ \acrshort{ci}-Tools für die Strategie ausgewählt, eine Projekt-Struktur und Pipeline definiert und
der eigentliche Shop inklusive angedachter Eigenentwicklungen implementiert.

\subsection{Projekthintergrund} \label{subsec:04-implementation-1}

Bei dem Shopware-Projekt, in welchem die\ \acrshort{ci}-Strategie implementiert werden soll, handelt es sich um einen
Anbieter von Kochutensilien.
Der Shop soll dabei auf der Basis von Shopware 6 neu entwickelt werden und das zuvor erstellte\ \acrshort{ci}-Konzept
in der Praxis einsetzen.
Das Ziel ist hierbei, durch die Integration der untersuchten \acrshort{ci}-Praktiken eine robuste, skalierbare und
effiziente Entwicklungsumgebung zu schaffen, die den Anforderungen moderner Online-Shops gerecht wird.
Dabei wird das zuvor ausgearbeitete \acrshort{ci}-Konzept als Leitfaden für die Entwicklung, Optimierung und
kontinuierliche Qualitätssicherung des Shops herangezogen.
Hierzu wird zunächst eine Shopware-Instanz aufgesetzt, welche anschließend über den Standardumfang der Software hinaus
um ein eigenes Theme und einige Plugins zur Anpassung verschiedener Bereiche des Shops ergänzt wird.
Die für den Shop angedachten Eigenentwicklungen umfassen folgende Plugins:

\begin{itemize}
    \item {
        \textbf{Eigene Produktkacheln auf Listing-Seiten}\par
        Bei Varianten-Artikeln wird keine direkte Auswahlmöglichkeit angeboten, stattdessen werden individuell
        gestaltete Produktkacheln präsentiert.
    }

    \item {
        \textbf{Eigener Produkt-Typ\ \glqq Rezept\grqq}\par
        Dieser spezielle Produkttyp kann vom Kunden betrachtet, aber nicht gekauft werden.
        Er dient zur Präsentation von Rezepten, die mit den im Shop erhältlichen Produkten zubereitet werden können.
    }
\end{itemize}

Als Produkt-Listing-Seiten werden Seiten der Shopware-Instanz bezeichnet, welche die Produkte einer bestimmten
Kategorie oder Gruppierung gesammelt in einer Liste anzeigen.
Produkte werden auf diesen Seiten im Standardumfang der Software als rechteckige Kacheln angezeigt, welche
verschiedene Informationen anzeigen, wie zum Beispiel eine Auswahl an Varianten des Artikels.
Durch das Klicken auf eine der Kacheln wird auf die jeweilige Produkt-Detail-Seite verwiesen.
Diese Seiten zeigen die Detail-Ansicht der Produkte auf und bieten weitere Informationen und Varianten- und
Kaufoptionen.
Für einige weitere benötigte Features des Shops wie eine Filialsuche und Social-Media-Feeds wurde der Einsatz von
Drittanbieter-Plugins eingeplant, um Entwicklungszeit einzusparen.

\subsection{Implementierung des Konzepts} \label{subsec:04-implementation-2}

Nachfolgend wird das als Fallstudie vorgesehene Shopware-Projekt anhand der untersuchten\ \acrshort{ci}-Praktiken
implementiert.
Zunächst wird gemäß dem Konzept eine Analyse der zur Verfügung stehenden\ \acrshort{ci}-Tools durchgeführt,
woraufhin Technologien für die Nutzung innerhalb des Projekts ausgewählt werden.
Anschließend wird die Projekt-Umgebung für die Nutzung von\ \acrshort{ci} aufgesetzt und das Shopware-Projekt
installiert und vorbereitet.
Daraufhin wird die Pipeline für das erstellte Projekt angelegt.
Hierbei werden die Phasen und Jobs erstellt und\ \acrshort{ci}-Tools konfiguriert, um automatisierte Tests und Analysen
durchführen zu können.
Nachdem das Projekt initialisiert wurde und die Pipeline funktionsfähig ist, wird die Implementierung der eigentlichen
Shop-Funktionen erläutert, wobei der Fokus auf die kontinuierliche Prüfung der Features durch die Pipeline gerichtet
ist.

\subsubsection{Auswahl von\ \acrshort{ci}-Tools}

Bevor mit der Implementierung des Projekts begonnen werden konnte, musste zunächst eine Auswahl an Tools getroffen
werden, die zur Umsetzung der \acrshort{ci}-Strategie eingesetzt werden können.
Hierfür wurden populäre Tools aus verschiedenen Bereichen, die in dem Projekt abgedeckt werden sollen, herangezogen:\\

\hspace{0.025\textwidth}%
\begin{minipage}{0.5\textwidth}
    \textbf{Version Control System}
    \vspace{-2mm}
    \begin{itemize}
        \setlength\itemsep{0em}
        \item Git / GitLab
        \item[]
    \end{itemize}
\end{minipage}%
\begin{minipage}{0.475\textwidth}
    \textbf{Pipeline-Runner}
    \vspace{-2mm}
    \begin{itemize}
        \setlength\itemsep{0em}
        \item GitLab CI/CD
        \item[]
    \end{itemize}
\end{minipage}

\hspace{0.025\textwidth}%
\begin{minipage}{0.5\textwidth}
    \textbf{PHP Testing Tools}
    \vspace{-2mm}
    \begin{itemize}
        \setlength\itemsep{0em}
        \item PHPUnit
        \item Infection
        \item[]
    \end{itemize}
\end{minipage}
\begin{minipage}{0.475\textwidth}
    \textbf{JavaScript Testing Tools}
    \vspace{-2mm}
    \begin{itemize}
        \setlength\itemsep{0em}
        \item Jest
        \item Cypress
        \item[]
    \end{itemize}
\end{minipage}

\hspace{0.025\textwidth}%
\begin{minipage}{0.5\textwidth}
    \textbf{PHP Static-Code-Analysis Tools}
    \vspace{-2mm}
    \begin{itemize}
        \setlength\itemsep{0.05em}
        \item PHP\_CodeSniffer
        \item PHP Mess Detector
        \item PHPStan
        \item Deptrac
        \item License Checker
        \item Security Checker
        \item[]
    \end{itemize}
\end{minipage}
\begin{minipage}{0.475\textwidth}
    \textbf{JavaScript Static-Code-Analysis Tools}
    \vspace{-2mm}
    \begin{itemize}
        \setlength\itemsep{-0.05em}
        \item Eslint
        \item Danger JS
        \item License Checker
        \item AuditJS
    \end{itemize}
    \textbf{Deployment Tools}
    \vspace{-2mm}
    \begin{itemize}
        \item Deployer
        \item[]
    \end{itemize}
    \vspace{-4.5mm}
\end{minipage}

Eine vollständige Übersicht und Beschreibung der eingesetzten \acrshort{ci}-Tools kann aus Anhang\ \ref{sec:appendix-1}
entnommen werden.

\subsubsection{Aufsetzen der Projekt-Umgebung}

Nach dem Festlegen der zu verwendenden Technologien wurde mit der Implementierung des Projekts begonnen, wobei
zuerst eine Umgebung für die Ausführung von Shopware und der gewählten\ \acrshort{ci}-Tools erstellt wurde.
Hierbei wurde sich für die Nutzung von Docker als Containerization-Technologie entschieden, da Shopware selbst ein
Docker-Image für das Betreiben der Plattform bereitstellt.\footpartcite{shopware-docker}
Auf der Basis des bereitgestellten Images wurde eine Shopware-Instanz aufgesetzt und anschließend eine Datenbank
angebunden.
Diese wurde, um in der\ \acrshort{ci}-Pipeline wiederverwendet werden zu können, auch aus einem Docker-Image
instanziiert.
Nach der Shopware-Installation wurden Grundkonfigurationen vorgenommen und erste manuelle Tests durchgeführt, um
sicherzustellen, dass die Instanz funktional ist.
Um die verschiedenen lokalen Container-Konfigurationen zentralisiert verwalten zu können, wurde beim Aufsetzen der
lokalen Umgebung das Tool\ \glqq Docker-Compose\grqq\ verwendet.
Dieses Tool eignet sich besonders für Multi-Container Docker-Applikationen, da in der Konfiguration hinterlegte
Container mit je einem Konsolenbefehl initialisiert, gestartet, gestoppt oder entfernt werden können.
\footpartcite{docker-compose}
Hierfür wurde eine Konfigurationsdatei im YAML-Format angelegt, welches zur Konfiguration von Programm-Daten
genutzt wird:

\yamlfile{code/docker-compose.yml}

Die aufgezeigte Konfigurationsdatei definiert zwei Container: \shellinline{shop} und \shellinline{database}.
Diese sind über ein angegebenes Netzwerk \shellinline{web} miteinander verbunden und haben jeweils offene Ports zur
Kommunikation hinterlegt.
Außerdem wurde das Docker-Volume \shellinline{database_volume} für das Persistieren der lokalen Datenbank angelegt,
welches sich um dessen Datenverwaltung kümmert.
Der Shop-Container nutzt dabei auch eine Volume-Anweisung in der Konfigurationsdatei, um die Projekt-Dateien und
Verzeichnisse in die Image-Instanz zu laden.
Darüber hinaus wurde für die Datenbank die zuvor angelegte Konfigurationsdatei \shellinline{.env} angegeben, in der
Umgebungsvariablen in Form von Tabellen-Konfiguration und Nutzerzugangsdaten hinterlegt werden konnten.
\\\\
Nachdem die lokale Entwicklungsumgebung erstellt wurde, konnte mit der Vorbereitung für die Umgebung der Pipelines
begonnen werden.
Hierzu wurde das gesamte Projekt in GitLab hinterlegt und ein Branching-Modell eingeführt, um sicherzustellen, dass
neue Features und Bugfixes vor der Integration in den Hauptzweig (\shellinline{main}) in isolierten Branches entwickelt
werden.
Neben dem\ \shellinline{main}-Branch, welcher als Deployment-Ziel die Produktionsumgebung hinterlegt hat, wurden auch
ein Development-Branch (\shellinline{development}) und -Server angelegt und eine Staging-Umgebung aufgesetzt.
Der Staging-Server ist dabei ein variables Deployment-Ziel, für welches die Auslieferung von
\shellinline{release}-Branches vorgesehen ist.
Die hierbei verwendete Branching-Strategie nennt sich\ \glqq Git-Flow\grqq.
Diese führt Namenskonventionen und verschiedene Arten von Branches ein, wobei neue Features und Anpassungen zunächst in
\shellinline{feature}-Branches entwickelt werden.
Anschließend werden diese in den\ \shellinline{development}-Branch gemerged, wo die Änderungen gesammelt werden.
Die gesammelten Features können daraufhin in einen\ \shellinline{release}-Branch überführt werden, in welchem keine
substanziellen Änderungen, sondern nur noch Fehlerbehebungen vorgenommen werden.
\shellinline{release}-Branches können zuletzt in den\ \shellinline{main}-Branch integriert werden.
Dieser stellt den Haupt-Versionsstand des Software-Projekts dar und speichert die aktuell ausgelieferte Version der
Software.
Neben diesen vier Arten von Branches gibt es noch die Möglichkeit,\ \shellinline{hoftix}-Branches für Fehlerbehebungen,
welche am Haupt-Versionsstand vorgenommen werden müssen.\footpartcite{git-flow}

\subsubsection{Erstellen der Pipeline}

Nachdem eine ausführende Umgebung definiert wurde, konnte die eigentliche Pipeline für das Projekt erstellt werden.
Diese soll zunächst bei jeder Integration in einen Branch ausgeführt werden und durchläuft dabei eine Build-,
Testing- und Deployment-Stage.
Im Build-Prozess werden die \acrshort{ci}-Tools und die Abhängigkeiten der Shopware-Platform installiert.
Der Testing-Prozess umfasst zunächst alle für die Strategie ausgewählten Testing- und \acrshort{qa}-Tools, inklusive
lang-andauernder Tests durch End-to-End-Testing.
Im späteren Verlauf der Entwicklung besteht durch Gitlab CI/CD die Möglichkeit, diese Tests nur in Branches mit
definierter Deployment-Umgebung durchführen zu lassen, falls dessen Laufzeit zu groß wird.
Abschließend liefern Jobs in der Deployment-Stage den entwickelten Code an die jeweilig für den Branch definierte
Umgebung aus.
\\\\
Die hierbei implementierten Phasen und Jobs der Pipeline wurden zur Ausführung in GitLab CI/CD in Form von YAML-Dateien
angelegt.
Erstellte Pipeline-Konfigurationsdateien werden dabei im Repository des Projekts hinterlegt, welches diese ausliest und
die Anweisungen an GitLab CI/CD weitergibt.
Somit konnten die einzelnen Stages der Pipeline und darin durchzuführende Jobs angelegt werden, sowie dessen
Ausführungsregeln, Beziehungen zueinander, anfallende Artifacts und Abhängigkeiten in Form von Services.
Nachfolgend wird ein kurzer Überblick der Implementierung einzelner Phasen der Pipeline in GitLab CI/CD gegeben:

\begin{itemize}
    \item {
        \textbf{Build-Stage}\par
        Die Build-Stage besteht aus einem einzelnen Build-Job, welcher die Installation der Abhängigkeiten
        von Shopware und der\ \acrshort{ci}-Tools durch Composer und \acrshort{npm} vornimmt.
        Im Folgenden wird die erstellte Konfiguration dieses Jobs für GitLab CI/CD aufgezeigt:

        \yamlfile{code/build.yml}\vspace{-2.3em}

        Die hierbei erzeugten Dateien werden als Artifacts für die Nutzung durch darauffolgende Jobs hinterlegt.
        Außerdem referenziert der Build-Job die Regel\ \glqq run-always\grqq, welche zuvor angelegt wurde und dafür
        sorgt, dass der Job bei jeder Integration in das Repository und auch bei Merge-Requests gestartet wird.
    }

    \item {
        \textbf{Testing-Stage}\par
        In der Testing-Stage werden die für das Projekt vorgesehenen Test- und \acrshort{qa}-Tools durchgeführt.
        Die Test-Jobs nutzen dabei die in der Build-Stage erstellten Artifacts, anstatt diese selbst erneut zu
        installieren, um die Ausführung der Pipeline insgesamt zu beschleunigen.
        Als Beispiel wird nachfolgend die Job-Konfiguration für das Unit-Testing-Tool \shellinline{PHPUnit} aufgezeigt:

        \yamlfile{code/test.yml}\vspace{-2.3em}

        Dieser Job gibt als Abhängigkeit den zuvor definierten Build-Job an und kann somit dessen installierte
        Artifacts nutzen.
        Da Unit-Tests in der Regel keine besonders zeitaufwändigen Tests darstellen, referenziert dieser Job auch
        die\ \glqq run-always\grqq-Regel.
        Außerdem generiert der Job eigene Artifacts in Form von Test-Reports, welche in GitLab CI/CD ausgewertet
        und in der jeweiligen Pipeline angezeigt werden.
        Tests, die weitere Abhängigkeiten voraussetzen, so wie Functional-Tests durch das Tool\ \shellinline{Cypress},
        folgen dabei dem gleichen Konfigurationsprinzip.
        Sie erweitern lediglich die Job-Konfiguration, um die benötigten Services, Umgebungsvariablen und weitere
        Abhängigkeiten, die spezifisch für den jeweiligen Test benötigt werden, zu installieren.
        Im weiteren Verlauf des Projekts können eigene spezielle Regeln für lang-andauernde Test-Jobs erstellt
        werden, sodass diese nur bei bestimmten Branches oder Aktionen im\ \acrshort{vcs} ausgeführt werden.
    }

    \item {
        \textbf{Deployment-Stage}\par
        Nach dem Erstellen der Build- und Testing-Stage wurde zuletzt die Deployment-Stage angelegt.
        Hierbei wurden verschiedene Jobs definiert, die jeweils zur Ausführung bei einer Integration in einen
        bestimmten Ziel-Branch konfiguriert wurden.
        Nachfolgend wird der Deployment-Job des\ \shellinline{main}-Branches aufgezeigt:

        \yamlfile{code/deploy.yml}\vspace{-2.3em}

        Der Job wurde so konfiguriert, dass die finale Ausführung zwar automatisiert abläuft, dabei aber manuell
        angestoßen werden muss, womit die umgesetzte Strategie effektiv\ \acrlong{cde} betreibt.
        Dies kann je nach den Bedürfnissen des Projekts angepasst werden und ermöglicht somit optional auch die
        Nutzung von\ \acrshort{cd} durch das vollautomatische Ausliefern nach dem erfolgreichen Durchlaufen der Build-
        und Testing-Jobs.
        Für den \shellinline{development}- und die \shellinline{release}-Branches, für welche auch
        Deployment-Umgebungen angelegt wurden, sind jeweils eigene Jobs definiert worden.
        Diese Jobs folgen dabei demselben Prinzip und nutzen das Tool \shellinline{Deployer}, welches das Ausliefern
        der Software an die jeweilige Umgebung anhand von eigens vorgegebenen Konfigurationen vornimmt.
        Die hierbei angelegten \shellinline{Deployer}-Konfigurationen für die verschiedenen Umgebungen wurden in Form
        von PHP-Code im Repository persistiert.
    }
\end{itemize}

Die grundlegenden Phasen und Jobs der Pipeline konnten somit anhand der erarbeiteten Strategie modelliert und
implementiert werden.
Sowohl der Build-Prozess als auch das automatisierte Testen und das Ausliefern des integrierten Codes wurden hierbei
mit einbezogen und umgesetzt.
Die Pipeline-Konfiguration bietet dabei eine gewisse Flexibilität und ermöglicht die Anpassung an verschiedene
Vorgaben und Anforderungen, welche sich im Verlauf des Projekts ändern können.
\\\\
In Abbildung\ \ref{fig:ci-pipeline-concept} wird die für das Projekt erstellte\ \acrshort{ci}-Pipeline visualisiert.
Hierbei werden verschiedene Arten von Branches in einem\ \acrshort{vcs} dargestellt, welche zunächst die gleiche Aktion
auslösen, den Build-Prozess.
Dort werden die \acrshort{ci}-Tools und die Abhängigkeiten von Shopware für die anschließende Nutzung in der
Testing-Stage installiert.
Die Testing-Stage wird nach erfolgreichem Durchlaufen der Build-Stage ausgeführt und ist in der Abbildung in zwei
Teile separiert.
Im oberen Teil werden die gewählten Testing- und \acrshort{qa}-Tools für JavaScript-Code der Strategie aufgezeigt,
während der untere Teil die Tools für PHP darstellt.
Wenn alle Jobs der Testing-Stage erfolgreich durchgelaufen sind, wird je nach Art des Ausgangs-Branches ein
Deployment-Prozess für die jeweilige Umgebung angestoßen.
Die Deployment-Stage wird nur für Branches mit definierter Umgebung ausgeführt und liefert dabei den Code des
Ausgangs-Branches an den Development-, Staging- oder Produktionsserver aus.

\begin{figure}[H]
    \centering
    \includegraphics[width=0.69420\textwidth]{images/content/ci-pipeline-concept}
    \captioncite[Eigene Darstellung]{}{Visualisierung der implementierten\ \acrshort{ci}-Pipeline}
    \label{fig:ci-pipeline-concept}
\end{figure}

Eine visuelle Ausgabe einzelner durchgeführter Pipelines des Projekts in GitLab CI/CD kann aus
Anhang\ \ref{sec:appendix-2} entnommen werden.

\subsubsection{Implementierung der Shop-Funktionen}

Durch die zuvor erstellten Umgebungen und die angelegte Pipeline-Konfiguration konnten nun die geplanten Funktionen des
Shops unter der Nutzung der erarbeiteten\ \acrshort{ci}-Praktiken implementiert werden.
Diese können grob in drei verschiedene Bereiche eingeteilt werden:

\begin{itemize}
    \item {
        \textbf{Entwicklung des Shop-Themes}\par
        Bei der Entwicklung des Themes für die Shopware-Instanz handelt es sich um viele kleine Anpassungen an dem
        bestehenden \acrfull{ui} des Standardumfangs.
        Da\ \acrshort{ui}-Anpassungen aufwendig zu testen sind und die Standard-\acrshort{ui} durch
        Shopwares eigene Tests abgedeckt ist, wurde das Theme ohne weitere Tests angelegt.
        Für Anpassungen, welche substanziell die Funktion von\ \acrshort{ui}-Komponenten verändern, können jedoch
        System-Tests angelegt werden, wobei geprüft wird, ob sich die angepasste\ \acrshort{ui} nach den gegebenen
        Vorschriften verhält.
        Dies hat den Vorteil, dass Bugs, die durch Updates der Standard-\acrshort{ui} mit bestehenden Anpassungen
        entstehen, schneller erkannt und gefixed werden können.
    }

    \clearpage

    \item {
        \textbf{Entwicklung des Plugins zur Anpassung der Produktkacheln}\par
        Um die angepassten Produktkacheln zu entwickeln, wurde ein eigenes Plugin angedacht, welches bestimmte
        Produkt-Daten in den Kacheln auf Produkt-Listing-Seiten des Shops anzeigt.
        Hier wird im Standard-Umfang der Shopware-Platform eine Auswahl an Varianten angeboten, welche durch das
        Plugin ersetzt werden soll.
        Statt der Variantenauswahl soll an dieser Stelle der Preis der günstigsten Variante angezeigt werden.
        Bei der Integration dieses Plugins konnten verschiedene Bereiche durch den Einsatz der Pipeline auf ihre
        Richtigkeit geprüft werden.
        Da die günstigste Artikelvariante zunächst Backend-seitig bestimmt und dem Frontend zur Verfügung
        gestellt werden musste, wurde bei der Entwicklung PHP-Code produziert.
        Für diesen konnten Unit-, Module- und Integration-Tests angelegt und dazugehörige\ \acrshort{ui}-Anpassungen
        im Frontend durch System-Tests abgedeckt werden.
        Die generelle Einhaltung gesetzter Code-Standards und weitere Qualitätsmerkmale wurden hierbei durch die
        gewählten statischen Code-Analyse-Tools überwacht.
    }

    \item {
        \textbf{Entwicklung des Plugins für Rezept-Produkte}\par
        Das Plugin für die Erstellung von Rezepten legt eine eigene Produkt-Art an, welche nicht aktiv im Shop gekauft
        werden kann und dabei lediglich die hinterlegten Rezept-Informationen anzeigt.
        Da dieses Feature sowohl Backend- als auch Frontend-Anpassungen benötigt, konnte der entwickelte PHP- und
        JavaScript-Code ebenfalls durch Unit-, Module- und Integration-Tests abdeckt werden.
        Die frontend-seitigen Erweiterungen bestimmter Produkt-Kacheln im Listing und der Produkt-Detail-Seite, welche
        nun Rezepte statt Produkten anzeigen, wurden dabei auch durch System-Tests abgedeckt.
        Vorteilhaft bei dieser Art der automatisierten Prüfung der erstellten Features ist, dass mit neuen
        Entwicklungen eingeführte Wechselwirkungen und Fehler schneller entdeckt werden können.
        So wurde zum Beispiel durch Regression- und System-Tests verhindert, dass das Anpassen der Produktkacheln
        durch beide erstellten Plugins zu unentdeckten Problemen nach der Integrationsphase führt.
    }
\end{itemize}

Bei der Entwicklung der Anpassungen wurden verschiedene\ \acrshort{ci}-Praktiken und Tools eingesetzt.
Der für die Anpassungen entwickelte PHP- und JavaScript-Code konnte hierbei durch automatisierte Unit-, Module- und
Integration-Tests der Tools\ \shellinline{PHPUnit} und\ \shellinline{Jest} abgedeckt werden.
Erstellte Unit-Tests in PHP wurden durch das \shellinline{Infection}-Framework mit Mutation-Tests kontinuierlich
evaluiert.
System-Tests konnten mit dem JavaScript-basierten Tool\ \shellinline{Cypress} erstellt und durchgeführt werden.
Durch\ \shellinline{Deptrac} konnte die Architektur der erstellten PHP-Klassen und Module geprüft werden.
Die Einhaltung gesetzter Coding-Standards und weitere Qualitätsmetriken konnten durch die Nutzung von
\shellinline{PHPCS}, \shellinline{PHPMD} und \shellinline{PHPStan} für PHP-Code und von \shellinline{Eslint} für
JavaScript-Code erhoben werden.
\shellinline{AuditJS} und der PHP\ \shellinline{Security Checker} konnten installierte Pakete automatisch auf
Sicherheitslücken überprüfen.
Um die Lizenzen der im Projekt installierten Pakete für das Projekt zu überwachen, wurden\ \shellinline{License Checker}
eingesetzt, welche die Lizenzen der von Composer und\ \acrshort{npm} installierten Pakete beobachten.
Außerdem wurde\ \shellinline{Danger JS} zur Steigerung der Transparenz des\ \acrshort{ci}-Prozesses eingesetzt, welches
den Status der einzelnen Jobs einer durchgelaufenen Pipeline in Form eines Kommentars im\ \acrshort{vcs} hinterlegt.
Letztlich wurden die Deployments für die jeweiligen Umgebungen durch das PHP-basierte Tool\ \shellinline{Deployer}
durchgeführt, wodurch die Ausfallzeiten beim Ausliefern der Software minimiert werden konnten.
\\
Drittanbieter-Plugins und Shop-Konfigurationen wurden hierbei nicht weiter getestet, da es sich bei ihnen nicht um
Eigenentwicklungen handelt und diese durch die jeweiligen Hersteller oder die Shopware-Platform selbst geprüft werden.
Des Weiteren wurden im Verlauf des Projekts neben den automatisierten Tests durch die\ \acrshort{ci}-Pipeline auch
manuelle Tests von Entwicklern, Projektleitern und den Kunden durchgeführt, um die System-Funktionalität
sicherzustellen.
Darüber hinaus wurde durch die Verwendung einer flexiblen Pipeline die Möglichkeit eingeräumt, das Projekt durch
Performance- und Last-Tests zu überwachen oder weitere Tools und Services zum Testen der Applikation anzubinden.
Dank des parallelen Ausführens der Test-Jobs konnte die Durchlaufzeit der Pipeline minimiert werden, wobei die
Strategie auch die Möglichkeit bietet, Jobs konditionell auszuführen, sollte deren Dauer die gesetzte Zehn-Minuten-Marke
übersteigen.

\clearpage

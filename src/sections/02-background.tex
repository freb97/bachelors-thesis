%! Author = Frederik Bußmann
%! Date = 22.06.2023

\section{Fachlicher Hintergrund} \label{sec:02-background}

Für die Erarbeitung einer geeigneten CI-Strategie werden zunächst die Begrifflichkeiten und Prinzipien von
Continuous-Integration definiert.
Darüber hinaus wird eine Übersicht über die Funktionen und Mechanismen der Shopware-Plattform gegeben.

\subsection{Begrifflichkeiten und Prinzipien von Continuous Integration} \label{subsec:02-background-1}

Das Kernprinzip hinter Continuous Integration (CI) wurde bereits im Jahr 1991 von Grady Booch definiert.
Hierbei werden Software-Releases nicht als ein großes Ereignis betrachtet, sondern regelmäßig durchgeführt, wobei
die vollständige Software stetig größer wird.
\footpartcite{booch}
Kent Beck popularisierte im Jahr 1998 die Disziplin des \glqq Extreme Programming\grqq, wobei großer Wert auf das frühe
und regelmäßige Testen und Integrieren der entwickelten Komponenten einer Software gelegt wird.
Beck behauptet hierbei, dass ein Feature, für welches es keine automatisierten Tests gibt, auch nicht funktioniert.
\footpartcite[2--4]{extreme-programming}
Im Jahr 2006 fasste Software-Entwickler Martin Fowler einige Bereiche dieser Methodiken in dem Artikel
\glqq Continuous Integration\grqq\ unter dem gleichnamigen Begriff zusammen.
Fowler beschreibt Continuous Integration als einen Prozess, bei dem Teammitglieder ihre Arbeit regelmäßig Integrieren,
wobei Integration als der Build-Prozess inklusive automatisierte Tests für die vollständige Software mitsamt der
erarbeiteten Änderungen zu verstehen ist.\footpartcite{fowler}
Nachfolgend werden einige wichtige Aspekte von Continuous Integration kurz dargestellt.

\subsubsection{Regelmäßige Integration}

Die namensgebende Methodik der CI ist das regelmäßige Integrieren von Software und damit Verbunden ist der
Software-Release.
In der traditionellen Softwareentwicklung wird der Release als ein einmaliges, großes Ereignis betrachtet.
Als Integration wird der Prozess des Einbindens einer einzeln entwickelten Komponente in die bisher bestehende
Gesamtheit einer Software bezeichnet, wobei der Release das Zusammenfinden aller Komponenten und das Ausführen des
Build-Prozesses bis hin zur fertigen, ausführbaren und auslieferbaren Software beschreibt.
In der Vergangenheit wurde bei einem Release jede Einzelkomponente der Software manuell integriert und getestet, wobei
dies als eigene Phase in der Entwicklung einer Applikation galt.
In einem Projekt mit CI

\subsubsection{Automatisierte Tests}

In der Regel wird Software vor einem Release getestet, wobei diese Tests in der Vergangenheit manuell durchgeführt
wurden.

\subsubsection{Reproduzierbarkeit}

\subsubsection{Feedback}

\subsection{Übersicht über die Shopware-Platform} \label{subsec:02-background-2}

Shopware wurde als Online-Shop-Software im Jahr 2000 durch Stefan Hamann ins Leben gerufen\footpartcite{shopware-story}
und bietet heute in ihrer aktuellen Major-Version 6 eine moderne E-Commerce-Plattform auf Basis des PHP-Frameworks
\glqq Symfony\grqq.
Das Symfony-Framework wird neben Shopware noch von anderen PHP-Basierten Projekten wie dem \gls{cms} \glqq Drupal\grqq,
dem Shop-System \glqq Magento\grqq\ und einigen weiteren Programmen\footpartcite{symfony-projects} als Grundlage genutzt
und bildet somit ein erprobtes Fundament für die Shopware-Plattform.
Shopware selbst ist nach der Installation bereits voll funktionsfähig und kann mit einem Backend und optional
mit einem Frontend oder für das Konsumieren der mitgelieferten \gls{api} eingerichtet werden.
Die Software kann auf verschiedenen Plattformen gehostet werden, darunter Linux-Server und containerisierte Umgebungen.
Darüber hinaus bietet Shopware als Unternehmen auch eine eigene Hosting-Lösung an, die speziell auf die Anforderungen
der Software zugeschnitten ist.
Die Plattform kann sowohl im Einzelbetrieb als auch als Cluster genutzt werden, um eine hohe Verfügbarkeit und
Skalierbarkeit zu gewährleisten.

\clearpage

%! Author = Frederik Bußmann
%! Date = 22.06.2023

\section{Evaluierung} \label{sec:05-evaluation}

Nach der Implementierung der \acrshort{ci}-Strategie im vorherigen Abschnitt folgt nun
die kritische Evaluierung des Konzepts anhand der entwickelten Fallstudie.
In diesem Kapitel wird die Effektivität und Anwendbarkeit der umgesetzten \acrshort{ci}-Strategie im Kontext des
erstellten Shopware-Projekts untersucht.
Dabei wird nicht nur die technische Umsetzung in den Blick genommen, sondern auch die praktischen Auswirkungen der
Strategie auf den Entwicklungsprozess und das Endprodukt.
Zunächst wird eine Analyse der Durchlaufzeit verwendeter Tools und durchgeführter Phasen der Strategie durchgeführt.
Dabei wird sowohl die lokale als auch die\ \acrshort{ci}-Umgebung betrachtet.
Anschließend wird anhand der durchgeführten Analyse eine Bewertung der Strategie vorgenommen, wobei Rückschlüsse auf
dessen Qualität und Effektivität gezogen, Erkenntnisse dargelegt und Auswirkungen diskutiert werden.

\subsection{Analyse und Vergleich} \label{subsec:05-evaluation-1}

Um die Effektivität der eingesetzten Strategie bewerten zu können, wurde zunächst eine Analyse der Durchlaufzeit
eingesetzter Tools durchgeführt.
Um eine vergleichbare Grundlage zu bieten, wurde die Zeit, die für die Durchführung einzelner Tools benötigt wird, je
dreimal lokal und in einer \acrshort{ci}-Pipeline gemessen und daraus die durchschnittliche Ausführungszeit ermittelt.

\begin{table}[H]
    \centering
    \begin{tabular}{|p{4cm}||p{4cm}|p{4cm}|}
        \hline
        Name des Tools & Durchlaufzeit \acrshort{ci} & Durchlaufzeit lokal                      \\
        \hline
        AuditJS               & \multicolumn{1}{r|}{$47s$}     & \multicolumn{1}{r|}{$19s$}     \\
        Cypress               & \multicolumn{1}{r|}{$2m\ 16s$} & \multicolumn{1}{r|}{$1m\ 32s$} \\
        Danger JS             & \multicolumn{1}{r|}{$1m\ 1s$}  & \multicolumn{1}{r|}{$-$}       \\
        Eslint                & \multicolumn{1}{r|}{$51s$}     & \multicolumn{1}{r|}{$12s$}     \\
        Jest                  & \multicolumn{1}{r|}{$54s$}     & \multicolumn{1}{r|}{$11s$}     \\
        (JS) License Checker  & \multicolumn{1}{r|}{$55s$}     & \multicolumn{1}{r|}{$14s$}     \\
        \hline
        Deptrac               & \multicolumn{1}{r|}{$36s$}     & \multicolumn{1}{r|}{$5s$}      \\
        Infection             & \multicolumn{1}{r|}{$1m\ 11s$} & \multicolumn{1}{r|}{$47s$}     \\
        \acrshort{phpcs}      & \multicolumn{1}{r|}{$44s$}     & \multicolumn{1}{r|}{$2s$}      \\
        \acrshort{phpmd}      & \multicolumn{1}{r|}{$42s$}     & \multicolumn{1}{r|}{$3s$}      \\
        PHPUnit               & \multicolumn{1}{r|}{$1m\ 7s$}  & \multicolumn{1}{r|}{$14s$}     \\
        PHPStan               & \multicolumn{1}{r|}{$38s$}     & \multicolumn{1}{r|}{$19s$}     \\
        PHP Security Checker  & \multicolumn{1}{r|}{$41s$}     & \multicolumn{1}{r|}{$9s$}      \\
        (PHP) License Checker & \multicolumn{1}{r|}{$44s$}     & \multicolumn{1}{r|}{$7s$}      \\
        \hline
        Deployer              & \multicolumn{1}{r|}{$5m\ 18s$} & \multicolumn{1}{r|}{$5m\ 58s$} \\
        \hline
    \end{tabular}
    \caption{Durchlaufzeit der Test-, \acrshort{qa}- und Deployment-Tools}
    \label{tab:ci-tools-time}
\end{table}

In Tabelle\ \ref{tab:ci-tools-time} werden die durchschnittlichen Zeiten für die Ausführung einzelner Test-,
\acrshort{qa}- und Deployment-Tools in der lokalen Entwicklungsumgebung und in der\ \acrshort{ci}-Pipeline aufgezeigt.
Es zeigt sich, dass das lokale Durchführen der Tools in der Regel schneller ist als in der\ \acrshort{ci}-Umgebung.
Dies lässt sich auf verschiedene Faktoren zurückschließen, wie der Mehraufwand durch das Initialisieren der einzelnen
Jobs und dessen Images im Pipeline-Runner, Wartezeiten durch die Speicherung von Job-Artifacts, Unterschiede in
der genutzten Hardware oder weiteren Umständen.
Da in der Tabelle nur die rohen Ausführungszeiten der Tools dargestellt werden, muss außerdem beachtet werden, dass bei
der manuellen lokalen Durchführung zwischen jedem Tool eine Unterbrechung stattfindet.
Dies ist dadurch bedingt, dass die testende Person die einzelnen Befehle zur Ausführung der Tools heraussuchen, in die
Kommandozeile einfügen und somit von Hand durchführen muss.
Da \shellinline{Danger JS} nur in einer\ \acrshort{ci}-Pipeline genutzt werden kann, wurde dieses bei den lokalen Tools
ausgelassen.
Bei der Ausführung der Tools in der Pipeline werden diese teils zeitgleich gestartet und laufen parallel, was die
Durchlaufzeit der Phasen der Pipeline insgesamt verkürzt.
Die Testing-Phase, welche Tools in Jobs parallel ausführt, dauert also nur so lange wie der langsamste Job.
Anhand der gemessenen Ausführungszeiten der Tools wurden anschließend Rückschlüsse auf die Durchlaufzeit der
Integrationsphasen des Projekts gezogen.

\begin{table}[H]
    \centering
    \begin{tabular}{|p{4cm}||p{4cm}|p{4cm}|}
        \hline
        Name der Phase & Durchlaufzeit \acrshort{ci} & Durchlaufzeit lokal         \\
        \hline
        Build  & \multicolumn{1}{r|}{$2m\ 12s$}  & \multicolumn{1}{r|}{$2m\ 47s$}  \\
        Test   & \multicolumn{1}{r|}{$2m\ 16s$}  & \multicolumn{1}{r|}{$4m\ 14s$}  \\
        Deploy & \multicolumn{1}{r|}{$5m\ 6s$}   & \multicolumn{1}{r|}{$5m\ 58s$}  \\
        \hline
        Gesamt & \multicolumn{1}{|r|}{$9m\ 34s$} & \multicolumn{1}{r|}{$12m\ 59s$} \\
        \hline
    \end{tabular}
    \caption{Dauer der Phasen einer Integration}
    \label{tab:integrations-time}
\end{table}

In Tabelle\ \ref{tab:integrations-time} werden die errechneten Durchlaufzeiten der Integrationsphasen des Projekts
aufgezeigt.
Die aufgelistete Zeit für die Build-Phase der\ \acrshort{ci}-Umgebung wurde hierbei aus dem Durchschnittswert von drei
in Pipelines ausgeführten Build-Jobs errechnet.
Für die Build-Phase in der lokalen Umgebung wurden dreimal die einzelnen Schritte des Vorgangs gemessen und
summiert, woraus anschließend auch der Durchschnitt ermittelt wurde.
Die Zeiten der Test- und Deploy-Phasen stützen sich auf die zuvor gemessenen Ausführungszeiten der im Projekt
verwendeten Tools.
Aus den Daten kann entnommen werden, dass eine Integration aus der lokalen Entwicklungsumgebung heraus theoretisch
nicht signifikant länger andauert als eine Integration durch die\ \acrshort{ci}-Umgebung.
Hierbei wird allerdings außer acht gelassen, dass die benötigte Zeit zur lokalen Ausführung des Build-Prozesses sowie
der Tests, \acrshort{qa}-Tools und der Deployment-Lösung in der Praxis viel höher sein kann.
Da Entwickler diese aktiv ausführen müssen, kommt es zwischen jedem Tool zu Verzögerungen.
Somit stellt die zur lokalen Durchführung errechnete Zeit gleichzeitig auch die bestmöglich zu erreichende Zeit für
den Integrationsprozess aus dieser Umgebung heraus dar.
Außerdem stützt sich die dargestellte Durchlaufzeit für lokale Deployments auf der Annahme, dass \shellinline{Deployer}
bereits vor der Implementierung der\ \acrshort{ci}-Strategie zur automatisierten Auslieferung genutzt wurde.
Ein vollständig manuell durchgeführtes Deployment würde hierbei wieder für Verzögerungen zwischen einzelnen Schritten
führen und die Durchlaufzeit der Phase erhöhen.
\\\\
Durch die Analyse-Tools von GitLab CI/CD konnte außerdem ermittelt werden, dass im Verlauf des Projekts insgesamt 132
Pipelines gestartet wurden, wovon 56 durch Fehler im Integrationsprozess abgebrochen wurden.
Die Fehlerrate von Integrationen in der durchgeführten Fallstudie beträgt somit circa $42\%$.
Diese hohe Fehlerzahl lässt sich zum Teil durch das häufige Fehlschlagen von Pipelines im frühen Entwicklungsstadium
des Projekts erklären.
Da anfangs die Erstellung einzelner Jobs und die ersten Konfigurationen von Test-, \acrshort{qa}- und
Deployment-Tools vorgenommen wurden, fielen in dieser Phase des Projekts vermehrt Fehler bei dessen Ausführung an.
\\
Insgesamt geben die in Tabelle\ \ref{tab:ci-tools-time} und\ \ref{tab:integrations-time} aufgeführten Zeiten
nur eine grobe Richtung vor, da verschiedene Aspekte wie die Qualität der vorhandenen Internetverbindung und
Hardwareressourcen, Wartezeiten von Jobs im Pipeline-Runner und weitere Faktoren die gemessenen Zeiten beeinflussen
können.
Sie können allerdings in Betrachtung der zuvor aufgeführten Umstände einen Indikator für die weitere Bewertung des
Erfolgs der Strategie bieten.
Die ermittelte Fehlerrate ausgeführter Integrationen ist dabei ein weiterer hilfreicher Anhaltspunkt.

\subsection{Erkenntnisse und Auswirkungen der Strategie} \label{subsec:05-evaluation-2}

Die Evaluierung der im vorherigen Abschnitt aufgezeigten Durchführungszeiten liefert einige Erkenntnisse über die
\acrshort{ci}-Strategie und hilft dabei, dessen Auswirkungen auf ein Projekt nachvollziehen zu können.
Es wird ersichtlich, dass die Nutzung einer\ \acrshort{ci}-Pipeline zur Automatisierung von Integrationen zumindest
einen kleinen zeitlichen Vorteil bietet.
Betrachtet man darüber hinaus den Aspekt, dass bei der manuellen Durchführung der einzelnen Schritte einer Integration
immer menschliche Verzögerungen und potenzielle Fehlerquellen hinzukommen, wird dieser Vorteil noch deutlicher.
Da Integrationen in modernen und agilen Softwareprojekten oft mehrmals stattfinden, summiert sich diese Zeitersparnis
mit jeder weiteren Durchführung.
\\\\
Durch die automatisierte und parallele Ausführung von Integrationsphasen in einer\ \acrshort{ci}-Pipeline konnte der
Entwicklungsprozess insgesamt beschleunigt werden.
Erstellte Features und Anpassungen konnten somit zügig und verlässlich integriert werden.
Außerdem verringerte dessen Einsatz den manuellen Aufwand des Integrationsprozesses, wodurch diese Zeit für andere
Aufgaben verwendet werden konnte.
Durch die Steigerung der Entwicklungsgeschwindigkeit unter der Verwendung der erstellten\ \acrshort{ci}-Strategie
konnte somit die geschäftsseitige Anforderung und Ziel der Arbeit\ \hyperlink{project-goals}{$Z_1$} erreicht werden.
\\\\
Neben der Beschleunigung des Entwicklungsprozesses führte die Verwendung der\ \acrshort{ci}-Strategie in der Fallstudie
zu der Findung von Fehlern beim Integrationsprozess.
Im Verlauf des Projekts wurden einige Pipelines aufgrund von Fehlern in der eingeführten Integration abgebrochen und
konnten somit die Entwickler des Projekts auf die zugrunde liegenden Probleme hinweisen.
Fehler und Inkonsistenzen konnten somit frühzeitig im Entwicklungszyklus erkannt und behoben werden.
Dieser Ansatz trug zur Erfüllung des Ziels\ \hyperlink{project-goals}{$Z_2$} bei, indem er die Anzahl an Fehlern, welche
die ausführenden Deployment-Umgebungen der Software und somit den Endkunden erreichen, mindert.
\\\\
Die\ \acrshort{ci}-Strategie hat außerdem die kontinuierliche Auslieferung neuer Softwareversionen erleichtert.
Durch die Automatisierung der Build- und Deployment-Prozesse konnten neue Features und Bugfixes schneller und
konsistenter an die Kunden ausgeliefert werden.
Hierdurch wurde die Fähigkeit des Entwicklungsteams gestärkt, auf sich ändernde Geschäftsanforderungen und
Marktbedingungen zu reagieren.
Die Strategie konnte somit auch zur Erreichung des Ziels\ \hyperlink{project-goals}{$Z_3$} beigetragen, indem sie eine
flexible und anpassbare Entwicklungsumgebung einführt, welche die kontinuierliche Weiterentwicklung und Auslieferung
von Software unterstützt.
\\\\
Zusammenfassend lässt sich sagen, dass die Implementierung der \acrshort{ci}-Strategie in der Fallstudie die
Effizienz und Qualität der entwickelten Software positiv beeinflusst hat.
Die Vorteile in Bezug auf Zeitersparnis, Fehlerreduktion und kontinuierliche Auslieferung haben dabei direkt zu der
Erreichung der zuvor definierten geschäftsseitigen Ziele $Z_n$ beigetragen und den Wert einer ganzheitlichen
\acrshort{ci}-Strategie weiter unterstrichen.

\clearpage

%! Author = Frederik Bußmann
%! Date = 22.06.2023

\section{Evaluierung} \label{sec:05-evaluation}

Nach der detaillierten Darstellung und Implementierung der \acrshort{ci}-Strategie im vorherigen Abschnitt, folgt nun
die kritische Evaluierung des Konzepts anhand der entwickelten Fallstudie.
In diesem Kapitel wird die Effektivität und Anwendbarkeit der umgesetzten \acrshort{ci}-Strategie im Kontext des
erstellten Shopware-Projekts untersucht.
Dabei wird nicht nur die technische Umsetzung in den Blick genommen, sondern auch die praktischen Auswirkungen der
Strategie auf den Entwicklungsprozess und das Endprodukt.

\subsection{Erkenntnisse und Auswirkungen der Strategie} \label{subsec:05-evaluation-1}

Die Umsetzung der \acrshort{ci}-Strategie in der durchgeführten Fallstudie hat gezeigt, dass das
primäre Ziel, eine qualitativ hochwertige Online-Shop-Lösung zu entwickeln, durch das Konzept erfüllt werden konnte.
Ein wesentlicher Grund dieser hohen Qualität war dabei die Implementierung umfangreicher automatisierter Tests.
Diese Tests konnten in der Fallstudie gewährleisten, dass ein Großteil der Softwarefehler und Probleme bereits vor der
Auslieferung an den Endnutzer identifiziert und behoben werden.
Hinderlich in der Nutzung der definierten\ \acrshort{ci}-Praktiken war hierbei das Implementieren komplexer
Test-Vorgänge und aufwändiger System-Tests, was allerdings auch zu einer erhöhten Test-Abdeckung führte.
Diese konnten im Verlauf des Projekts viele Fehler frühzeitig identifizieren und die Qualität der
entwickelten Software so wie die Interoperabilität verschiedener Komponenten überwachen.
Die vollständige Automatisierung der Pipeline verkürzte die Cycle Time, wodurch Änderungen insgesamt schneller an
den Projekt-Kunden ausgeliefert werden konnten.
Dieser beschleunigte Auslieferungsprozess ermöglicht es der Anwendung außerdem, sich schnell an neue Vorschriften und
Kundenanforderungen anzupassen.
Ein weiterer Vorteil der \acrshort{ci}-Strategie ergab sich durch den automatisierten Release-Prozess.
Ein störungsfreies und schnelles Deployment konnte verhindern, dass Kunden während der Nutzung der Software
Beeinträchtigungen erfahren.
Dies ist insbesondere im E-Commerce-Kontext von Bedeutung, da jede Unterbrechung oder Verzögerung zu einem monetären
Verlust führen kann.
\\\\
Die \acrshort{ci}-Strategie unterstütze durch den Einsatz von Containerization eine hohe Reproduzierbarkeit und
Konsistenz über verschiedene Umgebungen hinweg.
Hierdurch konnte sichergestellt werden, dass die in den Tests verwendete Anwendung nahezu identisch mit der später
ausgelieferten Produktions-Anwendung sind.
Darüber hinaus ermöglichte die Container-Technologie schnelle und zuverlässige Deployments, wodurch die
Cycle Time weiter reduziert werden konnte.
Die Transparenz der Strategie konnte durch die Nutzung von Tools zur kontinuierlichen Messung von Test-Ergebnissen
weiter gesteigert werden, da Entwickler so einen direkten Zugriff auf Pipeline-Ergebnisse und damit den Status der
erstellten Software erhielten.
Die Umsetzung der Strategie hat außerdem aufgezeigt, dass diese eine hohe Flexibilität aufweist und an die Bedürfnisse
eines individuellen Shopware-Projekts angepasst werden kann.
Zusammenfassend bestätigen die Ergebnisse die anfänglichen Hypothesen und zeigen die Vorteile der
\acrshort{ci}-Strategie sowohl aus wirtschaftlicher als auch aus technischer Sicht auf.
Insgesamt konnten die definierten geschäftsseitigen Ziele $Z_n$ der Arbeit durch den Einsatz der entwickelten Strategie
erreicht werden.

\clearpage

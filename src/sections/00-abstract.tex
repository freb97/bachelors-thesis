%! Author = Frederik Bußmann
%! Date = 22.06.2023

\section*{Abstract} \label{sec:00-abstract}

Das Ziel dieser Bachelorarbeit ist die Erarbeitung eines geeigneten Konzepts für das Einbinden von
Continuous-Development-Techniken in Shopware-Projekten.
Insbesondere die Praktiken und Methoden der Continuous Integration (\acrshort{ci}) werden im Verlauf der Arbeit
untersucht und erläutert.
Dabei wird die Bedeutung von \acrshort{ci} in der modernen Softwareentwicklung hervorgehoben, insbesondere im Kontext
von Shopware, eine der führenden E-Commerce-Plattformen in Deutschland.
Zunächst werden die Konzepte des Continuous Software Engineering vorgestellt und der fachliche Hintergrund für
weitere Themen und Aspekte der Arbeit definiert.
Die Arbeit beleuchtet außerdem die Herausforderungen und Vorteile, die mit der Implementierung von \acrshort{ci} in
Shopware-Projekten einhergehen.
Ein besonderer Fokus liegt auf der Automatisierung des Build-Prozesses, wodurch eine schnelle und effiziente
Integration von Softwarekomponenten ermöglicht wird.
Automatisierte Tests und \acrshort{qa}-Tools werden eingesetzt, um Standards vorzugeben und die Funktionalität der
Software sicherzustellen.
Darüber hinaus werden die positiven Auswirkungen von \acrshort{ci} auf die Entwicklungszeit von Software und die
allgemeine Softwarequalität diskutiert.
Neben der theoretischen Grundlagen wird ein praktischer Leitfaden für Entwickler und Unternehmen erstellt, um
\acrshort{ci}-Techniken in neue und bestehende Shopware-Projekte zu integrieren.
Dieser wird anschließend für die erstellte Strategie evaluiert und es werden einige Erkenntnisse und Schlussfolgerungen
dargelegt.
Insgesamt wird die Implementierung von Continuous Integration in Shopware-Projekten in der Arbeit untersucht,
durchgeführt und kritisch bewertet.

\clearpage

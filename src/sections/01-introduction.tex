%! Author = Frederik Bußmann
%! Date = 22.06.2023

\section{Einleitung} \label{sec:01-introduction}

Im Rahmen dieser Arbeit werden verschiedene Aspekte betrachtet, um ein Konzept für das Einbinden von
\acrfull{ci} in Shopware-basierten Projekten zu erarbeiten.
Shopware bietet als eine führende E-Commerce-Plattform und eine der bevorzugten Online-Shop-Lösungen in
Deutschland\footpartcite{shopware-usage-chart} eine solide Grundlage für Unternehmen, um im digitalen Raum
erfolgreich zu agieren.
Durch die gezielte Implementierung von \acrshort{ci}-Praktiken in solchen Projekten kann der Entwicklungszyklus
effizienter gestaltet und die Qualität des Endprodukts gesteigert werden.
Dies unterstützt Unternehmen dabei, ihre Wettbewerbsfähigkeit zu erhöhen und eine agile und reaktionsschnelle
Entwicklungsumgebung zu etablieren.

\subsection{Motivation} \label{subsec:01-introduction-1}

In der heutigen schnelllebigen digitalen Welt ist die Fähigkeit, qualitativ hochwertige Softwareprodukte schnell auf den
Markt zu bringen, nicht nur wünschenswert, sondern oft entscheidend für den Geschäftserfolg.
Die E-Commerce-Branche, geprägt durch ihre intensive Wettbewerbsdynamik, erfordert von Unternehmen eine kontinuierliche
Anpassung und Innovation, um im Markt bestehen zu können.
Hierbei nimmt die Effizienz und Effektivität der eingesetzten Softwareentwicklungsmethoden eine zentrale Rolle ein.
Um eine möglichst reaktionsschnelle und effektive Umgebung für Entwicklerteams in Shopware-basierten Projekten zu
schaffen, können Methodiken des Continuous-Software-Engineering verwendet werden.
Die Entwicklung einer robusten, jedoch flexiblen\ \acrshort{ci}-Strategie ist im Hinblick auf sich ständig
weiterentwickelnder Technologien und variierender Anforderungen besonders wichtig für die Sicherstellung der
Softwarequalität und wird in Zukunft immer relevanter.

\subsection{Zielsetzung} \label{subsec:01-introduction-2}

Die Entwicklung einer Continuous-Integration-Strategie für auf Shopware basierende Projekte ist das primäre Ziel
dieser Arbeit.
Die Strategie soll dazu beitragen, die Qualität der Software zu verbessern, die Effizienz des Entwicklungsprozesses zu
steigern und letztendlich die Kundenzufriedenheit zu erhöhen.
Die nachfolgend definierten Ziele $Z_n$ dienen als Leitfaden für die Konzeption und Entwicklung der
\acrshort{ci}-Strategie und stellen die geschäftsseitigen Anforderungen von Unternehmen in der E-Commerce-Branche an den
Entwicklungsprozess mit Shopware dar:

\begin{itemize} \hypertarget{project-goals}
    \item {
        \textbf{$Z_1$ Hohe Entwicklungsgeschwindigkeit}\par
        Die Einführung einer umfangreichen \acrshort{ci}-Strategie soll die Effizienz der Softwareentwicklungsteams
        verbessern und die Zeit bis zum Produkt-Release senken.
        Eine hohe Entwicklungsgeschwindigkeit sorgt für eine schnellere Auslieferung neuer Features und Fehlerbehebungen
        und somit zu einer niedrigeren Wartezeit für Kunden.
    }

    \item {
        \textbf{$Z_2$ Niedrige Fehlerrate}\par
        \acrshort{ci} soll dazu beitragen, Fehler frühzeitig im Entwicklungsprozess erkennen und beheben zu können, was
        die Qualität des Endprodukts verbessert.
        Die Stabilität und Qualität der ausgelieferten Anwendung wirkt sich durch weniger Ausfälle und eine
        geringere Supportzeit auf die Zufriedenheit von Kunden aus.
    }

    \item{
        \textbf{$Z_3$ Kontinuierliche Auslieferung neuer Software}\par
        Die \acrshort{ci}-Strategie und die damit verbundenen Prozesse, die sich für Entwicklerteams ergeben, sollen
        zu einer anpassbaren Entwicklungsumgebung führen.
        Diese Umgebung soll durch kontinuierliche Weiterentwicklung an die ständig wechselnden Anforderungen der
        modernen Softwareentwicklung angepasst werden können, was die Wettbewerbsfähigkeit fördert.
    }
\end{itemize}

Bei der Konzeption sollen diese Ziele verfolgt und die Maßnahmen der zu erarbeiteten \acrshort{ci}-Strategie
dementsprechend ergriffen werden.
Die Strategie soll dabei nicht nur die technischen Aspekte von \acrlong{ci} berücksichtigen, sondern auch die
organisatorischen Veränderungen, die mit der Einführung von\ \acrshort{ci} einhergehen.
Darüber hinaus soll die Strategie flexibel genug sein, um sich an zukünftige Veränderungen und Entwicklungen anpassen
zu können.

\subsection{Struktur der Arbeit} \label{subsec:01-introduction-3}

Im Laufe dieser Arbeit wird eine \acrshort{ci}-Strategie für Shopware-basierte Kundenprojekte konzeptioniert und
entwickelt.
Die Arbeit ist in fünf Hauptabschnitte unterteilt, die jeweils unterschiedliche Aspekte des Prozesses abdecken.

\subsubsection{Fachlicher Hintergrund}

In diesem Abschnitt wird zunächst der theoretische Rahmen für die Arbeit festgelegt.
Dies umfasst eine Einführung in das Continuous-Software-Engineering und die Prinzipien und Praktiken von
\acrlong{ci} sowie eine Übersicht über die Shopware-Platform.
Der Abschnitt dient dazu, ein grundlegendes Verständnis für die Themen und Technologien zu schaffen, die in
der Arbeit behandelt werden.

\subsubsection{Analyse und Konzept}

Dieser Abschnitt befasst sich mit der Analyse der aktuellen Situation und der Entwicklung eines Konzepts für die
\acrshort{ci}-Strategie.
Dies beinhaltet die Identifizierung von Herausforderungen und Anforderungen, die Berücksichtigung von Best
Practices und die Ausarbeitung eines Plans für die Implementierung der Strategie.
Die Konzeptionierung stützt sich dabei auf die im vorherigen Abschnitt aufgezeigte Fachliteratur.
Der Abschnitt dient als Brücke zwischen Theorie und Praxis und stellt sicher, dass die entwickelte Strategie sowohl
fundiert als auch anwendbar ist.

\subsubsection{Umsetzung der \acrshort{ci}-Strategie}

In diesem Abschnitt wird die Umsetzung der \acrshort{ci}-Strategie als Fallbeispiel beschrieben.
Dies umfasst die Auswahl und Konfiguration der benötigten Tools, die Definition von Prozessen und Workflows,
die Implementierung von Automatisierungen und Tests sowie das automatisierte Ausliefern der Software.
Der Schwerpunkt liegt hierbei auf der praktischen Umsetzung des zuvor entwickelten Konzepts und dessen Integration in
reale Shopware-Projekte.

\subsubsection{Evaluierung}

Die Auswertung der implementierten \acrshort{ci}-Strategie wird im folgenden Abschnitt behandelt.
Dabei wird die umgesetzte Strategie im Hinblick auf die im fachlichen Hintergrund aufgezeigten Prinzipien
geprüft.
Die Ergebnisse dieser Evaluierung werden analysiert und interpretiert, um Rückschlüsse auf den Erfolg der
Strategie zu ziehen.

\subsubsection{Schlussfolgerung und Ausblick}

Der letzte Abschnitt fasst die Ergebnisse der Arbeit zusammen und es werden Schlussfolgerungen über die
\acrshort{ci}-Strategie und dessen Anwendbarkeit in Shopware-Projekten gezogen.
Darüber hinaus wird ein Ausblick auf mögliche zukünftige Entwicklungen und Verbesserungen gegeben.
Dieser Abschnitt dient dazu, die Arbeit abzurunden und einen Ausblick auf weitere Forschungs- und
Entwicklungsarbeiten in diesem Bereich zu geben.

\clearpage

%! Author = Frederik Bußmann
%! Date = 22.06.2023

\section{Einleitung} \label{sec:01-introduction}

Im Rahmen dieser Arbeit werden verschiedene Aspekte betrachtet, um ein Konzept für das Einbinden von
Continuous Integration (\acrshort{ci}) in Shopware-basierten Projekten zu erarbeiten.
Shopware ist eine E-Commerce-Plattform und eine der meistgenutzten Online-Shop-Lösungen in
Deutschland\footpartcite{shopware-usage-chart}, welche es Unternehmen ermöglicht, ihre Präsenz im digitalen Markt
auszubauen.
Das Unternehmen best it GmbH hat für seine Kunden einige Shopware-Projekte im Einsatz, für welche im Nachfolgenden
eine \acrshort{ci}-Strategie zur automatisierten Prüfung und Auslieferung der entwickelten Software erstellt werden
soll.

\subsection{Unternehmensporträt der best it GmbH} \label{subsec:01-introduction-1}

Die best it GmbH ist eine Digitalagentur, die im Jahr 2000 von Manuel Strotmann gegründet wurde und sich auf die
Entwicklung von E-Commerce-Lösungen spezialisiert hat.
Die Firma ist in verschiedenen Bereichen des Online-Handels tätig und realisiert für ihre Kunden eine Vielzahl von
Shopware-Projekten.
Neben der Konzeption und Entwicklung von Online-Shops bietet das Unternehmen auch Design- und
Beratungsdienstleistungen für Kunden im E-Commerce-Bereich an.
Mittlerweile beschäftigt die best it GmbH über 100 Mitarbeiter an drei Standorten in Deutschland und Österreich.

\subsection{Zielsetzung} \label{subsec:01-introduction-2}

Das Ziel dieser Arbeit ist die Entwicklung einer Continuous-Integration-Strategie für auf Shopware basierende
Kundenprojekte der Firma best it.
Um eine Konzeption zu erstellen, die den Anforderungen des Unternehmens entspricht, werden die Ziele der Strategie
nachfolgend definiert:

\begin{itemize}
    \item \textbf{Höhere Entwicklungsgeschwindigkeit:} Die Einführung einer umfangreichen \acrshort{ci}-Strategie soll
          die Effizienz der Softwareentwicklungsteams verbessern und die Zeit bis zum Software-Release senken.

    \item \textbf{Frühe Fehlererkennung:} \acrshort{ci} soll dazu beitragen, Fehler frühzeitig im Entwicklungsprozess zu
          erkennen und zu beheben, was die Qualität des Endprodukts verbessert.

    \item \textbf{Bessere Kommunikation:} Die \acrshort{ci}-Strategie und die damit verbundenen Prozesse, die sich für
          Entwicklerteams ergeben, sollen zu einer verbesserten Kollaboration führen.
\end{itemize}

Bei der Konzeption sollen diese Ziele verfolgt und die Maßnahmen der zu erarbeiteten \acrshort{ci}-Strategie
dementsprechend ausgerichtet werden.
Die Strategie soll dabei nicht nur die technischen Aspekte von Continuous Integration berücksichtigen, sondern auch die
organisatorischen und kulturellen Veränderungen, die mit der Einführung von \acrshort{ci} einhergehen.
Darüber hinaus soll die Strategie flexibel genug sein, um sich an zukünftige Veränderungen und Entwicklungen anpassen
zu können.

\subsection{Struktur der Arbeit} \label{subsec:01-introduction-3}

Im Laufe dieser Arbeit wird eine \acrshort{ci}-Strategie konzeptioniert und entwickelt.
Die Arbeit ist in fünf Hauptabschnitte unterteilt, die jeweils unterschiedliche Aspekte des Prozesses abdecken.

\subsubsection{Fachlicher Hintergrund}

Zunächst wird der fachliche Hintergrund für die Arbeit festgelegt, der Abschnitt umfasst eine Einführung in das
Continuous-Software-Engineering und die Prinzipien und Praktiken von Continuous Integration, sowie eine Übersicht über die
Shopware-Plattform.
Dieser Abschnitt dient dazu, ein grundlegendes Verständnis für die Themen und Technologien zu schaffen, die in der
Arbeit behandelt werden.

\subsubsection{Analyse und Konzept}

Dieser Abschnitt befasst sich mit der Analyse der aktuellen Situation und der Entwicklung eines Konzepts für die
\acrshort{ci}-Strategie.
Dies beinhaltet die Identifizierung von Herausforderungen und Anforderungen, die Berücksichtigung von Best Practices
und die Ausarbeitung eines Plans für die Implementierung der Strategie.
Die Konzeptionierung stützt sich dabei auf die im vorherigen Abschnitt aufgezeigte Fachliteratur.

\subsubsection{Entwicklung der \acrshort{ci}-Strategie}

In diesem Abschnitt wird die Entwicklung der \acrshort{ci}-Strategie beschrieben.
Dies umfasst die Auswahl und Konfiguration der benötigten Tools, die Definition von Prozessen und Workflows und die
Implementierung von Automatisierungen und Tests.
Der Fokus liegt dabei auf der praktischen Anwendbarkeit des im vorherigen Abschnitt entwickelten Konzepts.

\subsubsection{Evaluierung}

Die Auswertung der implementierten \acrshort{ci}-Strategie wird im folgenden Abschnitt behandelt.
Dies beinhaltet die Durchführung von Tests und Messungen, um die Wirksamkeit und Effizienz der Strategie zu bewerten.
Dabei wird die umgesetzte Strategie im Hinblick auf die im fachlichen Hintergrund aufgezeigten Prinzipien geprüft.
Die Ergebnisse dieser Evaluierung werden analysiert und interpretiert, um Rückschlüsse auf den Erfolg der
Implementierung zu ziehen.

\subsubsection{Schlussfolgerung und Ausblick}

Der letzte Abschnitt fasst die Ergebnisse der Arbeit zusammen und es werden Schlussfolgerungen über die
\acrshort{ci}-Strategie und dessen Nutzung im Unternehmen gezogen.
Darüber hinaus wird ein Ausblick auf mögliche zukünftige Entwicklungen und Verbesserungen gegeben.
Dieser Abschnitt dient dazu, die Arbeit abzurunden und einen Ausblick auf weitere Forschungs- und Entwicklungsarbeiten
in diesem Bereich zu geben.

\clearpage

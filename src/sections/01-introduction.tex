%! Author = Frederik Bußmann
%! Date = 22.06.2023

\section{Einleitung} \label{sec:01-introduction}

Im Rahmen dieser Arbeit werden verschiedene Aspekte betrachtet, um ein Konzept für das Einbinden von
Continuous Integration (\acrshort{ci}) in Shopware-basierten Projekten zu erarbeiten.
Shopware ist eine E-Commerce-Plattform und eine der meistgenutzten Online-Shop-Lösungen in
Deutschland\footpartcite{shopware-usage-chart}, welche es Unternehmen ermöglicht, ihre Präsenz im digitalen Markt
auszubauen.
Die Implementierung von CI-Praktiken in solchen Projekten kann den Entwicklungszyklus beschleunigen und die
Qualität des Endprodukts erhöhen.
Dies trägt zur Stärkung der Wettbewerbsfähigkeit bei und fördert eine agile und reaktionsschnelle Entwicklungsumgebung.

\subsection{Motivation} \label{subsec:01-introduction-1}

In der heutigen schnelllebigen digitalen Welt ist die Fähigkeit, qualitativ hochwertige Softwareprodukte schnell auf den
Markt zu bringen, von entscheidender Bedeutung.
Die E-Commerce-Branche ist sehr wettbewerbsintensiv und Unternehmen müssen ständig innovieren, um relevant zu
bleiben.
Dabei spielt die Effizienz der genutzten Softwareentwicklungsmethoden eine wichtige Rolle.
Um eine möglichst reaktionsschnelle und effektive Umgebung für Entwicklerteams in Shopware-basierten Projekten zu
schaffen, können Methodiken des Continuous-Software-Engineering verwendet werden.
Die Entwicklung einer robusten, jedoch flexiblen\ \acrshort{ci}-Strategie ist im Hinblick auf sich ständig ändernde
Technologien und Anforderungen besonders wichtig für die Qualität von Software und wird in Zukunft immer relevanter.

\subsection{Zielsetzung} \label{subsec:01-introduction-2}

Die Entwicklung einer Continuous-Integration-Strategie für auf Shopware basierende Kundenprojekte ist das primäre Ziel
dieser Arbeit.
Die Strategie soll dazu beitragen, die Qualität der Software zu verbessern, die Effizienz des Entwicklungsprozesses zu
steigern und letztendlich die Kundenzufriedenheit zu erhöhen.
Die nachfolgend definierten Ziele $Z_n$ dienen als Leitfaden für die Entwicklung der CI-Strategie:

\begin{itemize}
    \item {
        \textbf{$Z_1$ Höhere Entwicklungsgeschwindigkeit:}\par
        Die Einführung einer umfangreichen \acrshort{ci}-Strategie soll durch die Automatisierung von Prozessen die
        Effizienz der Softwareentwicklungsteams verbessern und die Zeit bis zum Produkt-Release senken.
    }

    \item {
        \textbf{$Z_2$ Steigerung der Softwarequalität:}\par
        \acrshort{ci} soll aufgrund automatischer Tests und Analysen dazu beitragen, Fehler frühzeitig im
        Entwicklungsprozess erkennen und beheben zu können, was die Qualität des Endprodukts verbessert.
    }

    \item{
        \textbf{$Z_3$ Bessere Kommunikation:}\par
        Die \acrshort{ci}-Strategie und die damit verbundenen Prozesse die sich für Entwicklerteams ergeben, sollen
        zu einer Steigerung der Transparenz innerhalb des Teams und somit zu einer verbesserten Kollaboration
        führen.
    }
\end{itemize}

Bei der Konzeption sollen diese Ziele verfolgt und die Maßnahmen der zu erarbeiteten \acrshort{ci}-Strategie
dementsprechend ausgerichtet werden.
Die Strategie soll dabei nicht nur die technischen Aspekte von Continuous Integration berücksichtigen, sondern auch die
organisatorischen und kulturellen Veränderungen, die mit der Einführung von\ \acrshort{ci} einhergehen.
Darüber hinaus soll die Strategie flexibel genug sein, um sich an zukünftige Veränderungen und Entwicklungen anpassen
zu können.

\subsection{Struktur der Arbeit} \label{subsec:01-introduction-3}

Die Arbeit ist strukturell in die folgenden fünf Hauptabschnitte unterteilt:

\begin{itemize}
    \item {
        \textbf{Fachlicher Hintergrund}\par
        Zunächst wird der fachliche Hintergrund für die Arbeit festgelegt, der Abschnitt umfasst eine Einführung in das
        Continuous-Software-Engineering und die Prinzipien und Praktiken von Continuous Integration, sowie eine
        Übersicht über die Shopware-Plattform.
        Dieser Abschnitt dient dazu, ein grundlegendes Verständnis für die Themen und Technologien zu schaffen, die in
        der Arbeit behandelt werden.
    }

    \item {
        \textbf{Analyse und Konzept}\par
        Dieser Abschnitt befasst sich mit der Analyse der aktuellen Situation und der Entwicklung eines Konzepts für die
        \acrshort{ci}-Strategie.
        Dies beinhaltet die Identifizierung von Herausforderungen und Anforderungen, die Berücksichtigung von Best
        Practices und die Ausarbeitung eines Plans für die Implementierung der Strategie.
        Die Konzeptionierung stützt sich dabei auf die im vorherigen Abschnitt aufgezeigte Fachliteratur.
    }

    \item{
        \textbf{Entwicklung der \acrshort{ci}-Strategie}\par
        In diesem Abschnitt wird die Entwicklung der \acrshort{ci}-Strategie beschrieben.
        Dies umfasst die Auswahl und Konfiguration der benötigten Tools, die Definition von Prozessen und Workflows und
        die Implementierung von Automatisierungen und Tests.
        Der Fokus liegt dabei auf der praktischen Anwendbarkeit des im vorherigen Abschnitt entwickelten Konzepts.
    }

    \item{
        \textbf{Evaluierung}\par
        Die Auswertung der implementierten \acrshort{ci}-Strategie wird im folgenden Abschnitt behandelt.
        Dies beinhaltet die Durchführung von Tests und Messungen, um die Wirksamkeit und Effizienz der Strategie zu
        bewerten.
        Dabei wird die umgesetzte Strategie im Hinblick auf die im fachlichen Hintergrund aufgezeigten Prinzipien
        geprüft.
        Die Ergebnisse dieser Evaluierung werden analysiert und interpretiert, um Rückschlüsse auf den Erfolg der
        Implementierung zu ziehen.
    }

    \item{
        \textbf{Schlussfolgerung und Ausblick}\par
        Der letzte Abschnitt fasst die Ergebnisse der Arbeit zusammen und es werden Schlussfolgerungen über die
        \acrshort{ci}-Strategie und dessen Nutzung in Shopware-Projekten gezogen.
        Darüber hinaus wird ein Ausblick auf mögliche zukünftige Entwicklungen und Verbesserungen gegeben.
        Dieser Abschnitt dient dazu, die Arbeit abzurunden und einen Ausblick auf weitere Forschungs- und
        Entwicklungsarbeiten in diesem Bereich zu geben.
    }
\end{itemize}

\clearpage

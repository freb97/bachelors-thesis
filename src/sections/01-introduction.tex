%! Author = Frederik Bußmann
%! Date = 22.06.2023

\section{Einleitung} \label{sec:01-introduction}

Im Rahmen dieser Arbeit werden verschiedene Aspekte betrachtet, um ein Konzept für das Einbinden von
Continuous Integration (CI) in Shopware-basierten Projekten zu erarbeiten.
Shopware ist eine E-Commerce-Plattform und eine der meistgenutzten Online-Shop-Lösungen in
Deutschland\footpartcite{shopware-usage-chart}, welche es Unternehmen ermöglicht, ihre Präsenz im digitalen Markt
auszubauen.
Das Unternehmen best it GmbH hat für seine Kunden einige Shopware-Projekte im Einsatz, für welche im Nachfolgenden
eine CI-Strategie zur automatisierten Prüfung und Auslieferung der entwickelten Software erstellt werden soll.

\subsection{Unternehmensporträt der best it GmbH} \label{subsec:01-introduction-1}

Die best it GmbH ist eine Digitalagentur, die im Jahr 2000 von Manuel Strotmann gegründet wurde und sich auf die
Entwicklung von E-Commerce-Lösungen spezialisiert hat.
Die Firma ist in verschiedenen Bereichen des Online-Handels tätig und realisiert für ihre Kunden eine Vielzahl von
Shopware-Projekten.
Neben der Konzeption und Entwicklung von Online-Shops bietet das Unternehmen auch Design- und
Beratungsdienstleistungen für Kunden im E-Commerce-Bereich an.
Mittlerweile beschäftigt die best it GmbH über 100 Mitarbeiter an drei Standorten in Deutschland und Österreich.

\subsection{Zielsetzung} \label{subsec:01-introduction-2}

Das Ziel dieser Arbeit ist die Entwicklung einer Continuous-Integration-Strategie für auf Shopware basierende
Kundenprojekte der Firma best it.
Um eine Konzeption zu erstellen, die den Anforderungen des Unternehmens entspricht, werden die Ziele der Strategie
nachfolgend definiert:

\begin{itemize}
    \item \textbf{Höhere Entwicklungsgeschwindigkeit:} Die Einführung einer umfangreichen CI-Strategie soll die
          Effizienz der Softwareentwicklungsteams verbessern und die Zeit bis zum Software-Release senken.

    \item \textbf{Frühe Fehlererkennung:} CI soll dazu beitragen, Fehler frühzeitig im Entwicklungsprozess zu erkennen
          und zu beheben, was die Qualität des Endprodukts verbessert.

    \item \textbf{Bessere Kommunikation:} Die CI-Strategie und die damit verbundenen Prozesse, die sich für
          Entwicklerteams ergeben, sollen zu einer verbesserten Kollaboration führen.
\end{itemize}

Bei der Konzeption sollen diese Ziele verfolgt und die Maßnahmen der zu erarbeiteten CI-Strategie dementsprechend
ausgerichtet werden.
Die Strategie soll dabei nicht nur die technischen Aspekte von Continuous Integration berücksichtigen, sondern auch die
organisatorischen und kulturellen Veränderungen, die mit der Einführung von CI einhergehen.
Darüber hinaus soll die Strategie flexibel genug sein, um sich an zukünftige Veränderungen und Entwicklungen anpassen
zu können.

\subsection{Struktur der Arbeit} \label{subsec:01-introduction-3}

\clearpage

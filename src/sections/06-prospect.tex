%! Author = Frederik Bußmann
%! Date = 22.06.2023

\section{Schlussfolgerungen und Ausblick} \label{sec:06-prospect}

% @TODO: Text verlängern, verbessern

Im folgenden, abschließenden Kapitel werden Schlussfolgerungen zur konzipierten\ \acrshort{ci}-Strategie gezogen
und ein Fazit aufgestellt.
Letztlich wird ein Ausblick über mögliche zukünftige Erweiterungen und Anpassungen der Strategie gegeben.

\subsection{Fazit} \label{subsec:06-prospect-1}

% @TODO: Text verlängern und ein wenig umschreiben

Die Entwicklung und Implementierung einer \acrshort{ci}-Strategie ist ein komplexer Prozess, der sowohl technische als
auch organisatorische Herausforderungen mit sich bringt.
Die konzipierte \acrshort{ci}-Strategie hat sich als effektiv und in der Praxis anwendbar erwiesen.
Durch die Automatisierung von Build-, Test- und Deployment-Prozessen konnte die Qualität des Endprodukts signifikant
gesteigert werden.
Fehler wurden frühzeitig erkannt und behoben, wodurch die Auslieferung an den Endnutzer reibungsloser und zuverlässiger
wurde.
Die Implementierung umfangreicher automatisierter Tests hat nicht nur dazu beigetragen, die Softwarequalität zu erhöhen,
sondern auch das Vertrauen des Entwicklerteams in den eigenen Code gestärkt.
Die Transparenz, die durch die \acrshort{ci}-Strategie geschaffen wurde, hat die Zusammenarbeit im Team verbessert und
den Entwicklungsprozess beschleunigt.
Es wurde jedoch auch deutlich, dass die Einführung einer solchen Strategie eine sorgfältige Planung und Anpassung an die
spezifischen Anforderungen des Projekts erfordert.
Nicht jede \acrshort{ci}-Lösung ist für jedes Projekt geeignet, und es ist wichtig, die richtigen Tools und Prozesse
für die jeweilige Aufgabe auszuwählen.

\subsection{Ausblick} \label{subsec:06-prospect-2}

% @TODO: Text zu Ende schreiben und verbessern

In der Zukunft könnten weitere Optimierungen und Erweiterungen der \acrshort{ci}-Strategie in Betracht gezogen werden.
Eine mögliche Erweiterung wäre , um .
Darüber hinaus könnten weitere Teststrategien, wie beispielsweise Performance- oder Sicherheitstests, in die
Pipeline integriert werden, um die Qualität des Endprodukts weiter zu erhöhen.
Außerdem könnte die \acrshort{ci}-Strategie in anderen technologischen Kontexten oder für größere Projekte evaluiert
werden, um ihre Skalierbarkeit und Anpassungsfähigkeit zu überprüfen.


\clearpage

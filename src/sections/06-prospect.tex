%! Author = Frederik Bußmann
%! Date = 22.06.2023

\section{Schlussfolgerungen und Ausblick} \label{sec:06-prospect}

Im folgenden, abschließenden Kapitel werden die zentralen Erkenntnisse und Resultate dieser Arbeit reflektiert.
Es werden Schlussfolgerungen zur konzipierten\ \acrshort{ci}-Strategie aufgestellt und ein Fazit zur erstellten
Arbeit gezogen, wobei der Fokus auf der praktischen Umsetzung, den erreichten Zielen und den Herausforderungen liegt,
die während der Implementierung aufgetreten sind.
Letztlich wird ein Ausblick über mögliche zukünftige Erweiterungen und Anpassungen der Strategie gegeben, um diese noch
effizienter und anpassungsfähiger im Hinblick auf zukünftige Projekte und dessen Anforderungen zu gestalten.

\subsection{Fazit} \label{subsec:06-prospect-1}

Die Entwicklung und Implementierung einer \acrshort{ci}-Strategie ist ein komplexer Prozess, der sowohl technische als
auch organisatorische Herausforderungen mit sich bringt.
Die konzipierte \acrshort{ci}-Strategie, die im Laufe dieser Arbeit entwickelt und evaluiert wurde, hat gezeigt, wie
effektiv der Einsatz von Continuous-Development-Techniken in modernen Softwareprojekten sein kann.
Durch die Automatisierung von Build-, Test- und Deployment-Prozessen konnte die Qualität des Endprodukts signifikant
gesteigert werden.
Fehler wurden frühzeitig erkannt und behoben, wodurch die Auslieferung an den Endnutzer reibungsloser und zuverlässiger
wurde.
Die Implementierung umfangreicher automatisierter Tests hat nicht nur dazu beigetragen, die Softwarequalität zu erhöhen,
sondern auch das Vertrauen des Entwicklerteams in den eigenen Code gestärkt.
Die Transparenz, die durch die \acrshort{ci}-Strategie geschaffen wurde, hat die Zusammenarbeit im Team verbessert und
den Entwicklungsprozess beschleunigt.
Es wurde jedoch auch deutlich, dass die Einführung einer solchen Strategie eine sorgfältige Planung und Anpassung an die
spezifischen Anforderungen des Projekts erfordert und die Komplexität des Entwicklungsprozesses steigert.
Nicht jede \acrshort{ci}-Lösung ist für jedes Projekt geeignet, und es ist wichtig, die richtigen Tools und Prozesse
für die jeweilige Aufgabe auszuwählen.
\\\\
Abschließend lässt sich sagen, dass die \acrshort{ci}-Strategie trotz ihrer Komplexität und der damit verbundenen
Herausforderungen einen positiven Einfluss auf die Entwicklung von Shopware-Projekten und dessen Effizienz und Qualität
haben kann.
Sie ermöglicht es den Entwicklerteams, agil auf Veränderungen zu reagieren, den Entwicklungsprozess zu optimieren
und gleichzeitig die Qualität des Endprodukts zu gewährleisten.
Die kontinuierliche Integration und Auslieferung fördert zudem eine Kultur der ständigen Verbesserung, in der
Feedbackschleifen verkürzt und Innovationen schneller umgesetzt werden können.
Die in dieser Arbeit vorgestellte \acrshort{ci}-Strategie ist ein Beispiel dafür, wie solch ein Ansatz in der Praxis
umgesetzt und optimiert werden kann.
Es ist jedoch wichtig zu betonen, dass die kontinuierliche Integration und Auslieferung kein starres Konzept ist,
sondern vielmehr ein dynamischer Prozess, der ständige Anpassung und Weiterentwicklung erfordert.
Die Technologielandschaft, die Anforderungen der Stakeholder und die Bedürfnisse des Marktes ändern sich ständig, und
es ist die Aufgabe des Entwicklerteams, darauf flexibel zu reagieren und die \acrshort{ci}-Strategie entsprechend
anzupassen.
Insgesamt bietet die erstellte \acrshort{ci}-Strategie nicht nur technische Vorteile, sondern auch wirtschaftliche und
organisatorische.
Sie unterstützt Unternehmen dabei, wettbewerbsfähig zu bleiben und Kundenanforderungen an Shopware-Projekte effizient
zu erfüllen.

\subsection{Ausblick} \label{subsec:06-prospect-2}

Die in der Arbeit erstellte\ \acrshort{ci}-Strategie deckt einige wichtige Aspekte der modernen Softwareentwicklung
ab.
Hierbei bieten sich dennoch einige Möglichkeiten für zukünftige Erweiterungen und Optimierungen.
Ein zentrales Thema, das in zukünftigen Arbeiten weiter vertieft werden könnte, ist die Integration des
\glqq DevOps\grqq-Konzepts.
Das Prinzip entsprang aus der steigenden Kluft zwischen der Entwicklung und dem Betreiben von Software in großen
Unternehmen.
Es befasst sich mit dem Vereinheitlichen des Prozesses der Softwareentwicklung und dessen Auslieferung an
dafür vorgesehene Umgebungen.\footpartcite[178]{fitzgerald}
Die Arbeit beschränkte sich auf das Konzipieren reproduzierbarer Entwicklungs- und Pipeline-Umgebungen.
Hierbei könnte die weitere Vereinheitlichung der ausführenden Deployment-Infrastruktur eine nahtlose Integration
zwischen Entwicklung, Testing und Produktion ermöglichen.
Dies würde nicht nur die lokale Entwicklungsumgebung und die Pipelines betreffen, sondern auch die
Produktionsumgebung, wodurch die genutzte Infrastruktur des Projekts ganzheitlich abgedeckt würde.
\\\\
Ein weiterer interessanter Ansatzpunkt ist die Evaluierung der \acrshort{ci}-Strategie in anderen technologischen
Kontexten oder für größere Shopware-Projekte.
Dies würde wertvolle Erkenntnisse über die Skalierbarkeit und Anpassungsfähigkeit des erstellten Konzepts liefern.
Shopware bietet die Möglichkeit, in einem Cluster betrieben zu werden.\footpartcite{shopware-cluster}
Hierbei werden mehrere Instanzen der Software gleichzeitig im Produktionsbetrieb angewandt, oftmals an
verschiedenen Orten, um eine erhöhte Verfügbarkeit zu bieten und die Zugriffsgeschwindigkeit für Nutzer über große
Flächen sicherzustellen.
In Zukunft könnte die Strategie dazu für das Ausführen verteilter Systeme untersucht und erweitert werden.
\\\\
Insgesamt zeigt sich, dass die Evolution von CI/CD-Strategien in der modernen Softwareentwicklung von zentraler
Bedeutung ist.
Mit der stetigen Weiterentwicklung von Technologien und Methoden wird die Bedeutung einer robusten und flexiblen
CI/CD-Strategie in den kommenden Jahren weiter zunehmen, um den Anforderungen von Unternehmen und Entwicklern gerecht
zu werden.

\clearpage

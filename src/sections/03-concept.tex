%! Author = Frederik Bußmann
%! Date = 22.06.2023

\section{Analyse und Konzept} \label{sec:03-concept}

Im folgenden Kapitel werden verschiedene Techniken zur kontinuierlichen Integrierung von Code analysiert und eine
Konzeption für Shopware-basierte Projekte erarbeitet.

Um Kundenprojekte mit Shopware umsetzen zu können, muss zunächst definiert sein, welche Bereiche der Plattform
anpassbar sind und wie diese Anpassungen zustande kommen.
Da das Shop-Backend und Frontend für Kunden erweiterbar sein soll, werden sogenannte \glqq Plugins\grqq\ verwendet.
Bei Plugins handelt es sich um Erweiterungen in Form von Code, welche einer durch die Plattform vorgegeben Ordner-
und Datei-Struktur folgt.
Hierbei können sowohl die für verschiedene Backend-Bereiche zuständigen PHP-Dateien als auch die für das Back- und
Frontend verwendeten Templates, JavaScript-Logik und Style-Dateien erweitert werden.
Für die Analyse der kontinuierlichen Integrierung der Kundenprojekte wird sich somit im folgenden auf die von Shopware
vorgegebenen Erweiterungsmöglichkeiten bezogen.

\clearpage

%! Author = Frederik Bußmann
%! Date = 21.10.2021

% Preamble
%! Suppress = FileNotFound
\documentclass[ngerman,11pt]{article}

% Packages
\usepackage[T1]{fontenc}
\usepackage[utf8]{inputenc}
\usepackage{babel}
\usepackage{csquotes}
\usepackage[nottoc]{tocbibind}
\usepackage[outputdir=/build/temp]{minted}
\usepackage[backend=biber,style=numeric,citestyle=ieee,autocite=superscript]{biblatex}

\usepackage[colorlinks,
    urlcolor = blue,
    linkcolor = black,
    citecolor = black,
    pdfpagelabels,
    pdfstartview = FitH,
    bookmarksopen = true,
    bookmarksnumbered = true,
    plainpages = false,
    hypertexnames = false] {hyperref}

\usepackage{listings}
\usepackage{setspace}
\usepackage{amsmath}
\usepackage{graphicx}
\usepackage{tikz}
\usepackage{geometry}
\usepackage{afterpage}
\usepackage{glossaries}
\usepackage{float}

\usepackage[justification=centering]{caption}
\usepackage[noabbrev,nameinlink]{cleveref}
\usepackage[acronym, nonumberlist, toc, section=section]{glossaries-extra}

% Full-page background pictures
\usepackage{eso-pic}

% Decorations over headline
\usepackage{scrhack}
\usepackage[automark,headsepline]{scrlayer-scrpage}

% PDF Info
\hypersetup{
    pdfinfo={
        Author={Frederik Bußmann},
        Title={Konzeption und Entwicklung einer Continuous-Integration-Strategie für Kundenprojekte auf Basis der
        Shopware-Platform},
        Subject={Bachelor-Arbeit, Leitung Prof.\ Dr.-Ing.\ Martin Schulten},
        Keywords={},
    }
}
\title{Konzeption und Entwicklung einer Continuous-Integration-Strategie für Kundenprojekte auf Basis der
Shopware-Platform}
\author{Frederik Bußmann}
\date{Juli 2023}

% Blank page command
\newcommand\blankpage{\null\thispagestyle{empty}\newpage}

% Background picture command
\newcommand\BackgroundPic{%
\put(0,0){%
\parbox[b][\paperheight]{\paperwidth}{%
    \vfill
    \centering
    \includegraphics[width=\paperwidth,height=\paperheight,%
        keepaspectratio]{images/layout/background}%
    \vfill
}}}

% Settings
\setstretch{1.25}
\setlength{\parindent}{0pt}
\graphicspath{{/data}}
\addbibresource{config/references.bib}
\crefname{figure}{Abbildung}{Abbildung}
\Crefname{figure}{Abbildung}{Abbildung}

% Glossary Settings
\makeglossaries
\glstoctrue
%! Author = Frederik Bußmann
%! Date = 22.06.2023

\newacronym{ci}{CI}{Continuous Integration}

\newacronym{cd}{CD}{Continuous Deployment}

\newacronym{cde}{CDE}{Continuous Delivery}

\newacronym{ce}{CE}{Continuous Exploration}

\newacronym{ide}{IDE}{Integrated Development Environment}

\newacronym{api}{API}{Application Programming Interface}

\newacronym{cms}{CMS}{Content Management System}

\newacronym{vcs}{VCS}{Version Control System}

\newacronym{di}{DI}{Dependency Injection}

\newacronym{orm}{ORM}{Object-Relational Mapping}

\newacronym{npm}{NPM}{Node Package Manager}

\newacronym{yaml}{YAML}{YAML Ain't Markup Language}

\newacronym{ms}{MS}{Mutation Score}

\newacronym{msi}{MSI}{Mutation Score Indicator}

\newacronym{pr}{PR}{Pull Request}

\newacronym{mr}{MR}{Merge Request}

\newacronym{css}{CSS}{Cascading Style Sheets}

\newacronym{qa}{QA}{Quality Assurance}

\newacronym{cli}{CLI}{Command-Line Interface}

\newacronym{phpmd}{PHPMD}{PHP Mess Detector}

\newacronym{phpcs}{PHPCS}{PHP Code Sniffer}


% Minted Configuration
%! Author = Frederik Bußmann
%! Date = 23.06.2023

\renewcommand\listingscaption{Code-Abbildung}
\renewcommand\listoflistingscaption{Quellcodeverzeichnis}

\crefname{listing}{Code-Abbildung}{Code-Abbildungen}
\Crefname{listing}{Code-Abbildung}{Code-Abbildungen}

\renewcommand{\listoflistings}{%
    \phantomsection
    \addcontentsline{toc}{section}{\listoflistingscaption}%
    \listof{listing}{\listoflistingscaption}%
}

\newmintedfile[jsfile]{javascript}{
    bgcolor=black!10,
    linenos=true,
    numberblanklines=true,
    numbersep=5pt,
    gobble=0,
    frame=leftline,
    framerule=0.4pt,
    framesep=2mm,
    funcnamehighlighting=true,
    tabsize=4,
    obeytabs=false,
    mathescape=false,
    showspaces=false,
    showtabs =false,
    texcl=false,
}

\newmintinline[jsinline]{javascript}{
    bgcolor=black!10,
}

\newmintinline[phpinline]{php}{
    bgcolor=black!10,
}

\newmintinline[shellinline]{shell}{
    bgcolor=black!10,
}


% Document
\begin{document}

    % Title page
    \AddToShipoutPicture{\BackgroundPic}
    \pagenumbering{gobble}
    %! Author = Frederik Bußmann
%! Date = 01.08.2023

\begin{titlepage}
    \newgeometry{left=3.6cm,right=3.6cm,top=2cm,bottom=2.5cm}
        \begin{tikzpicture}
            \begin{scope}
                \node{\includegraphics[width=0.5\textwidth]{images/layout/w-hs}};
            \end{scope}
            \begin{scope}[xshift=8.4cm,yshift=0.38cm]
                \node{\includegraphics[width=0.35\textwidth]{images/layout/w-hs-text}};
            \end{scope}
            \label{fig:title}
        \end{tikzpicture}
        \vspace{1.3cm}

        \begingroup
        \fontsize{44pt}{46pt}\selectfont
        {\bfseries Bachelorarbeit}
        \endgroup

        \vskip 1.44cm

        \begingroup
        \fontsize{8pt}{6pt}\selectfont
        Titel der Arbeit // Title of Thesis
        \endgroup

        \begingroup
        \fontsize{12pt}{14pt}\selectfont
        {\bfseries Konzeption und Entwicklung einer Continuous-Integration-Strategie für Kundenprojekte auf Basis der
        Shopware-Platform\par}
        \endgroup
        \vskip -0.1cm

        \noindent\rule{14.4cm}{0.4pt}

        \vskip 0.05cm

        \begingroup
        \fontsize{8pt}{6pt}\selectfont
        Akademischer Abschlussgrad: Grad, Fachrichtung (Abkürzung) // Degree
        \endgroup

        \vskip -0.05cm

        \begingroup
        \fontsize{12pt}{14pt}\selectfont
        Bachelor of Science (B.Sc.)
        \endgroup
        \vskip -0.1cm

        \noindent\rule{14.4cm}{0.4pt}

        \vskip 0.05cm

        \begingroup
        \fontsize{8pt}{6pt}\selectfont
        Autorenname, Geburtsort // Name, Place of Birth
        \endgroup

        \vskip -0.05cm

        \begingroup
        \fontsize{12pt}{14pt}\selectfont
        Frederik Bußmann, Coesfeld
        \endgroup
        \vskip -0.1cm

        \noindent\rule{14.4cm}{0.4pt}

        \vskip 0.05cm

        \begingroup
        \fontsize{8pt}{6pt}\selectfont
        Studiengang // Course of Study
        \endgroup

        \vskip -0.05cm

        \begingroup
        \fontsize{12pt}{14pt}\selectfont
        Informatik.Softwaresysteme
        \endgroup
        \vskip -0.1cm

        \noindent\rule{14.4cm}{0.4pt}

        \vskip 0.05cm

        \begingroup
        \fontsize{8pt}{6pt}\selectfont
        Fachbereich // Department
        \endgroup

        \vskip -0.05cm

        \begingroup
        \fontsize{12pt}{14pt}\selectfont
        Wirtschaft und Informationstechnik
        \endgroup
        \vskip -0.1cm

        \noindent\rule{14.4cm}{0.4pt}

        \vskip 0.05cm

        \begingroup
        \fontsize{8pt}{6pt}\selectfont
        Erstprüferin/Erstprüfer // First Examiner
        \endgroup

        \vskip -0.05cm

        \begingroup
        \fontsize{12pt}{14pt}\selectfont
        Prof.\ Dr.-Ing.\ Martin Schulten
        \endgroup
        \vskip -0.1cm

        \noindent\rule{14.4cm}{0.4pt}

        \vskip 0.05cm

        \begingroup
        \fontsize{8pt}{6pt}\selectfont
        Zweitprüferin/Zweitprüfer // Second Examiner
        \endgroup

        \vskip -0.05cm

        \begingroup
        \fontsize{12pt}{14pt}\selectfont
        Martin Knoop
        \endgroup
        \vskip -0.1cm

        \noindent\rule{14.4cm}{0.4pt}

        \vskip 0.05cm

        \begingroup
        \fontsize{8pt}{6pt}\selectfont
        Abgabedatum // Date of Submission
        \endgroup

        \vskip -0.05cm

        \begingroup
        \fontsize{12pt}{14pt}\selectfont
        01.08.2023
        \endgroup
        \vskip -0.1cm

        \noindent\rule{14.4cm}{0.4pt}
    \restoregeometry
\end{titlepage}
\clearpage


    % Statement
    %! Author = Frederik Bußmann
%! Date = 22.06.2023

\thispagestyle{empty}
\newgeometry{left=3.6cm,right=3.6cm,top=8.8cm,bottom=2.5cm}
    \begingroup
        \fontsize{18pt}{20pt}\selectfont
        {\bfseries Eidesstattliche Versicherung}
    \endgroup

    \vskip 0.8cm

    \begingroup
        \fontsize{12pt}{18pt}\selectfont
        Bußmann, Frederik
    \endgroup

    \vskip -0.35cm

    \noindent\rule{14.4cm}{0.4pt}

    \vskip -0.2cm

    \begingroup
        \fontsize{8pt}{6pt}\selectfont
        Name, Vorname // Name, First Name
    \endgroup

    \vskip 0.6cm
    \begingroup
        \fontsize{10.5pt}{11.5pt}\selectfont
        Ich versichere hiermit an Eides statt, dass ich die vorliegende Abschlussarbeit mit dem Titel
    \endgroup

    \vskip 0.3cm

    \begingroup
        \fontsize{12pt}{18pt}\selectfont
        {\bfseries Konzeption und Entwicklung einer Continuous-Integration-Strategie für Kundenprojekte auf Basis der
        Shopware-Platform}
    \endgroup

    \vskip 0.3cm

    \begingroup
        \fontsize{10.5pt}{11.5pt}\selectfont
        selbstständig und ohne unzulässige fremde Hilfe erbracht habe.
        Ich habe keine anderen als die angegebenen Quellen und Hilfsmittel benutzt sowie wörtliche und sinngemäße
        Zitate kenntlich gemacht.
        Die Arbeit hat in gleicher oder ähnlicher Form noch keiner Prüfungsbehörde vorgelegen.
    \endgroup

    \vskip 0.8cm
    {\fontsize{12pt}{18pt}\selectfont
    Stadtlohn, den}

    \vskip -0.35cm

    \noindent\rule{14.4cm}{0.4pt}

    \vskip -0.2cm

    \begingroup
        \fontsize{8pt}{6pt}\selectfont
        Ort, Datum, Unterschrift // Place, Date, Signature
    \endgroup
\restoregeometry
\clearpage

    \ClearShipoutPicture

    % Empty page
    \afterpage{\blankpage}
    \clearpage

    % Abstract
    \thispagestyle{empty}
    %! Author = Frederik Bußmann
%! Date = 01.08.2023

\section*{Abstract} \label{sec:00-abstract}

Lorem ipsum dolor sit amet.

\clearpage


    % Table of contents
    \tableofcontents
    \clearpage

    % Document content
    \pagenumbering{arabic}
    \setcounter{page}{1}
    %! Author = Frederik Bußmann
%! Date = 01.08.2023

\section{Einleitung} \label{sec:01-introduction}

Lorem ipsum dolor sit amet~\footpartcite{westermann}.

\clearpage


    % Bibliography
    \pagenumbering{Roman}
    \printbibliography[heading=bibintoc, title={Literaturverzeichnis}]
    \clearpage

    % List of figures
    \listoffigures
    \clearpage

    % List of source code
    \listoflistings
    \clearpage

    % Glossary
    %\glsaddall
    %\printglossary
    %\clearpage

    % Acronyms
    %\addtocontents{toc}{\protect\setcounter{tocdepth}{0}}
    %\printglossary[type=\acronymtype,title=Abkürzungsverzeichnis]
    %\addtocontents{toc}{\protect\setcounter{tocdepth}{2}}
    %\clearpage

    % Appendix
    %\setcounter{secnumdepth}{0}
    %%! Author = Frederik Bußmann
%! Date = 22.06.2023

\section{Anhang I: Lokale Docker-Konfiguration} \label{sec:appendix-1}

Im folgenden wird die in Kapitel\ \ref{subsec:04-implementation-1} aufgeführte Datei\ \shellinline{docker-compose.yml}
dargestellt, welche in der erarbeiteten \acrshort{ci}-Strategie für das Verwalten lokaler Container-Services zum
Ausführen von Shopware verwendet wird:

\yamlfile{code/docker-compose.yml}

\clearpage

    %\clearpage
\end{document}
